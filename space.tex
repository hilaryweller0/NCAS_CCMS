
\chapter{Spatial Discritisation for Atmospheric Modelling}

\section*{Aim}
\begin{itemize}
\item Describe a multitude of spatial discretisation methods used in atmospheric
modelling
\item Describe their advantages and disadvantages
\item What modelling centres use what methods
\item Different meshes of the sphere
\end{itemize}

\section*{Methods}

\begin{multicols}{2}
\begin{itemize}
\item Finite difference
\item Arakawa grids 
\item Semi-Lagrangian
\item Finite Volume
\item Finite Element
\item Spectral Method
\item Spectral Element
\item Mixed finite element
\item Discontinuous Galerkin
\end{itemize}
\end{multicols}

\textcolor{red}{No spatial discretisation method is perfect. But some
produce useful forecasts.}

\clearpage{}

\section*{Shallow Water Equations}

\begin{minipage}[t]{0.48\columnwidth}%
\textbf{Non-linear}
\begin{align}
\frac{D\mathbf{u}}{Dt} & =-2\Omega\times\mathbf{u}-g\nabla(h+h_{0})\label{eqn:SWEuv}\\
\frac{Dh}{Dt} & +h\nabla\cdot\mathbf{u}=0\label{eqn:SWEh}
\end{align}
$\Omega$ is rotation

\textbf{Linearise}

$\mathbf{u}$ is small

$h\rightarrow H+h$ where $H$ is mean height and $h$ is small (ignore
mountain)\optBlock{
\begin{align*}
\frac{\partial\mathbf{u}}{\partial t} & =-2\Omega\times\mathbf{u}-g\nabla h\\
\frac{\partial h}{\partial t} & +H\nabla\cdot\mathbf{u}=0
\end{align*}
}%
\end{minipage} %
\begin{minipage}[t]{0.48\columnwidth}%
{\small{}Solutions of non-linear SWE }{\small\par}

{\small{}\mediaMovie[autostart,loop]
{\includegraphics[width=0.9\linewidth]{/home/hilary/OpenFOAM/hilary-dev/run/shallowWaterSphere/WilliMountain/RinglerLloyd/6/2/129600/hU.pdf}}
{/home/hilary/OpenFOAM/hilary-dev/run/shallowWaterSphere/WilliMountain/RinglerLloyd/6/2/hU.mp4}}{\small\par}

{\small{}Colour shows height of atmosphere.}{\small\par}

{\small{}Vectors show depth integrated wind.}{\small\par}%
\end{minipage}

\clearpage{}

\section*{Finite Differences - Arakawa A-grid}

Store discrete values of $h$ and $u$ on a grid. A one dimensional
example:

\begin{minipage}[t]{0.6\columnwidth}%
\phantom{}\resizebox{1\textwidth}{!}{\input{/home/hilary/latex/teaching/topics/SWE/figs/A-grid.pdftex_t}}%
\end{minipage}%
\begin{minipage}[t]{0.38\columnwidth}%
\begin{eqnarray*}
\frac{\partial u}{\partial x}_{j} & \approx & \optBlock{\frac{u_{j+1}-u_{j-1}}{2\Delta x}}\\
\frac{\partial h}{\partial x}_{j} & \approx & \optBlock{\frac{h_{j+1}-h_{j-1}}{2\Delta x}}
\end{eqnarray*}
%
\end{minipage}\stepwise{
\switch{\includegraphics[width=0.3\linewidth]{/home/hilary/latex/teaching/topics/SWE/2dSWE/Agrid1.pdf}}
{\movie*{10}{1}{\includegraphics[width=0.3\linewidth]{/home/hilary/latex/teaching/topics/SWE/2dSWE/Agrid\thesubstep.pdf}}}
}

The Arakawa A-grid is not used by operational centres in this form.

\clearpage{}

\section*{Arakawa Grids \citep{AL77}}

Consider the component form of the 2D linearised SWE:
\begin{eqnarray}
\frac{\partial u}{\partial t} & = & \phantom{-}fv-g\frac{\partial h}{\partial x}\label{eq:linSWEu}\\
\frac{\partial v}{\partial t} & = & -fu-g\frac{\partial h}{\partial y}\label{eq:linSWEv}\\
\frac{\partial h}{\partial t} & = & -H\left(\frac{\partial u}{\partial x}+\frac{\partial v}{\partial y}\right)\label{eq:linSWEh}
\end{eqnarray}
\begin{minipage}[t]{0.28\columnwidth}%
\begin{center}
\textbf{A-grid}
\par\end{center}
\ifthenelse{\boolean{@lecversion}}%
{\scalebox{0.7}[0.7]{\input{/home/hilary/latex/teaching/topics/SWE/Arakawa/A.pdf_t}}}
{\scalebox{0.7}[0.7]{\input{/home/hilary/latex/teaching/topics/SWE/Arakawa/grid.pdf_t}}}%
\end{minipage} %
\begin{minipage}[t]{0.7\columnwidth}%
\phantom{}
\begin{itemize}
\item Non-compact discretisation of $\partial u/\partial x$, $\partial u/\partial y$,
$\partial h/\partial x$ and $\partial h/\partial y$ leads to spurious
gravity modes. 
\item Grid-scale oscillations 
\item \optBlock{Simple}
\item \optBlock{Acurate representation of inertial oscillations ($u$ and $v$ are together)}
\end{itemize}
%
\end{minipage}

\begin{minipage}[t]{0.28\columnwidth}%
\begin{center}
\textbf{B-grid}
\par\end{center}
\ifthenelse{\boolean{@lecversion}}%
{\scalebox{0.7}[0.7]{\input{/home/hilary/latex/teaching/topics/SWE/Arakawa/B.pdf_t}}}
{\scalebox{0.7}[0.7]{\input{/home/hilary/latex/teaching/topics/SWE/Arakawa/grid.pdf_t}}}%
\end{minipage} %
\begin{minipage}[t]{0.7\columnwidth}%
\phantom{}
\begin{itemize}
\item Some older ocean models use B-grid. Eg ocean component of HadCM3.
\item Gradients are compact but averaging needed
\item \optBlock{Still computational modes}
\item \optBlock{Acurate representation of inertial oscillations ($u$ and $v$ are together)}
\end{itemize}
%
\end{minipage}

\begin{minipage}[t]{0.28\columnwidth}%
\begin{center}
\textbf{C-grid}
\par\end{center}
\ifthenelse{\boolean{@lecversion}}%
{\scalebox{0.7}[0.7]{\input{/home/hilary/latex/teaching/topics/SWE/Arakawa/C.pdf_t}}}
{\scalebox{0.7}[0.7]{\input{/home/hilary/latex/teaching/topics/SWE/Arakawa/grid.pdf_t}}}%
\end{minipage} %
\begin{minipage}[t]{0.7\columnwidth}%
\phantom{}
\begin{itemize}
\item Compact discretisation of \optBlock{$\partial u/\partial x$, $\partial u/\partial y$,
$\partial h/\partial x$ and $\partial h/\partial y$}\\
$\rightarrow$ best representation of gravity modes. 
\item Coriolis terms and geostrophic balance more of a problem.
\item Widely used including current UK Met Office model.
\end{itemize}
%
\end{minipage}

\begin{minipage}[t]{0.28\columnwidth}%
\begin{center}
\textbf{D-grid}
\par\end{center}
\ifthenelse{\boolean{@lecversion}}%
{\scalebox{0.7}[0.7]{\input{/home/hilary/latex/teaching/topics/SWE/Arakawa/D.pdf_t}}}
{\scalebox{0.7}[0.7]{\input{/home/hilary/latex/teaching/topics/SWE/Arakawa/grid.pdf_t}}}%
\end{minipage} %
\begin{minipage}[t]{0.7\columnwidth}%
\phantom{}
\begin{itemize}
\item Non-compact discretisation of $\partial u/\partial x$ and $\partial h/\partial x$
leads to spurious gravity modes 
\item Lots of averaging needed
\item Best representation of geostrophic balance
\item GFDL's FV3 model uses a C-D grid
\end{itemize}
%
\end{minipage}

\begin{minipage}[t]{0.28\columnwidth}%
\begin{center}
\textbf{E-grid}
\par\end{center}
\ifthenelse{\boolean{@lecversion}}%
{\scalebox{0.7}[0.7]{\input{/home/hilary/latex/teaching/topics/SWE/Arakawa/E.pdf_t}}}
{\scalebox{0.7}[0.7]{\input{/home/hilary/latex/teaching/topics/SWE/Arakawa/grid.pdf_t}}}%
\end{minipage} %
\begin{minipage}[t]{0.7\columnwidth}%
\phantom{}
\begin{itemize}
\item Equivalent to a B-grid rotated by $45^{o}$
\end{itemize}
%
\end{minipage}

\clearpage{}

\section*{Solutions of Linearised SWE starting from an initial bump}

\stepwise{
\begin{tabular}{ccc}
A-grid & B-grid & C-grid \\
\switch{\includegraphics[width=0.3\linewidth]{/home/hilary/latex/teaching/topics/SWE/2dSWE/Agrid1.pdf}}
{\movie*{10}{1}{\includegraphics[width=0.3\linewidth]{/home/hilary/latex/teaching/topics/SWE/2dSWE/Agrid\thesubstep.pdf}}}
&
\switch{\includegraphics[width=0.3\linewidth]{/home/hilary/latex/teaching/topics/SWE/2dSWE/Bgrid1.pdf}}
{\movie*{10}{1}{\includegraphics[width=0.3\linewidth]{/home/hilary/latex/teaching/topics/SWE/2dSWE/Bgrid\thesubstep.pdf}}}
&
\switch{\includegraphics[width=0.3\linewidth]{/home/hilary/latex/teaching/topics/SWE/2dSWE/Cgrid1.pdf}}
{\movie*{10}{1}{\includegraphics[width=0.3\linewidth]{/home/hilary/latex/teaching/topics/SWE/2dSWE/Cgrid\thesubstep.pdf}}}
\end{tabular}
}

\clearpage{}

\begin{minipage}[t]{0.4\columnwidth}%

\section*{The Pole Problem}

\includegraphics[width=1\textwidth]{/home/hilary/latex/teaching/topics/grids/latLon}%
\end{minipage}\hfill{}%
\begin{minipage}[t]{0.55\columnwidth}%
\textbf{Convergence of points towards the poles}
\begin{itemize}
\item Severe time-step restrictions
\item Parallel scaling bottlenecks
\end{itemize}
\bigskip{}

\textbf{Who uses a lat-lon grid?}
\begin{itemize}
\item UK Met Office (moving to cubed sphere)
\item NOAA (moving to cubed sphere)
\item Environment Canada (for low resolution. Yin-Yang for high resolution)
\item NCAR Community Atmosphere Model CAM-FV (for low resolution. CAM-SE
on cubed sphere for high resolution)
\end{itemize}
%
\end{minipage}

\clearpage{}

\begin{minipage}[t]{0.48\columnwidth}%

\section*{Cubed Sphere}

\includegraphics[width=1\textwidth]{graphics/a_cubedsphere}

Grid lines are non-orthogonal which must be treated accurately to
avoid grid imprinting.%
\end{minipage} \hfill{}%
\begin{minipage}[t]{0.47\columnwidth}%

\subsubsection*{Used by}

\optBlock{
\begin{itemize}
\item Next UK Met Office model
\begin{itemize}
\item Mixed finite elements
\end{itemize}
\item GFDL's FV3 (will also be used by NOAA)
\begin{itemize}
\item finite volume, CD grid
\end{itemize}
\item NCAR CAM-SE
\begin{itemize}
\item spectral element method
\end{itemize}
\item NUMA-NEPTUNE, from the US Navy (not operatation)
\begin{itemize}
\item discontinuous Galerkin
\end{itemize}
\end{itemize}
}%
\end{minipage}

\clearpage{}

\begin{minipage}[t]{0.48\columnwidth}%

\section*{Reduced grid}

\includegraphics[width=1\textwidth]{/home/hilary/latex/teaching/topics/grids/reduced}%
\end{minipage} \hfill{}%
\begin{minipage}[t]{0.47\columnwidth}%

\subsubsection*{Used by}
\begin{itemize}
\item ECMWF IFS
\begin{itemize}
\item spectral method
\end{itemize}
\item ECMWF FV3
\begin{itemize}
\item experimental finite volume model using A-grid
\end{itemize}
\end{itemize}
%
\end{minipage}

\clearpage{}

\begin{minipage}[t][0.5\paperheight]{0.48\columnwidth}%

\section*{Icosahedral grids}

Suitable for low order models

\includegraphics[width=1\textwidth]{/home/hilary/latex/teaching/topics/grids/hexIcos}

\vspace{-3cm}
\includegraphics[width=1\textwidth]{/home/hilary/latex/teaching/topics/grids/triIcos}%
\end{minipage} \hfill{}%
\begin{minipage}[t]{0.47\columnwidth}%
\textbf{Used by}

\optBlock{
\begin{itemize}
\item NICAM
\begin{itemize}
\item Japanese A-grid model run at very high resolution
\end{itemize}
\item MPAS (Voronoi, US community model to replace WRF)
\begin{itemize}
\item finite volume C-grid
\end{itemize}
\item ICON (triangles, German operational model)
\begin{itemize}
\item finite volume C-grid
\end{itemize}
\item DYNAMICO (French experimental model)
\begin{itemize}
\item finite volume C-grid
\end{itemize}
\item Colorado State University Model (Uses Vorticity is a prognostic variable,
Z-grid)
\end{itemize}
}%
\end{minipage}

\clearpage{}

\begin{minipage}[t][0.5\paperheight]{0.48\columnwidth}%

\section*{Yin-Yang Grid}

\includegraphics[width=1\textwidth]{/home/hilary/latex/teaching/topics/grids/grid_yin_yang}

Conservation is problematic

Spurious behaviour on overlaps%
\end{minipage} \hfill{}%
\begin{minipage}[t]{0.47\columnwidth}%

\subsubsection*{Used by}
\begin{itemize}
\item Environment Canada
\begin{itemize}
\item for high resolution modelling
\item semi-implicit, semi-Lagrangian (like UK Met Office)
\end{itemize}
\end{itemize}
%
\end{minipage}

\clearpage{}

\section*{Discretisations for terms of the Euler Equations}

\begin{tabular}{ll}
\textcolor{blue}{Linear Advection}\hspace{1cm} & \textcolor{cyan}{Non-linear advection}\tabularnewline
\multicolumn{2}{l}{\textcolor{red}{Terms involved in gravity and acoustic wave propagation}}\tabularnewline
\multicolumn{2}{l}{\textcolor{olive}{Coriolis}}\tabularnewline
\end{tabular}

{\global\long\def\arraystretch{0.8}

\begin{tabular}{ll}
Momentum  & $\frac{\partial\mathbf{u}}{\partial t}+{\color{cyan}\mathbf{u}\cdot\nabla\mathbf{u}}=\text{-}{\color{olive}2\bm{\Omega}\times\mathbf{u}-{\color{red}\frac{\nabla p}{\rho}+\mathbf{g}}}$ \tabularnewline
 & \tabularnewline
Continuity  & $\frac{\partial\rho}{\partial t}+{\color{red}\nabla\cdot\rho\mathbf{u}}=0$\tabularnewline
 & \tabularnewline
Potential temperature  & $\frac{\partial\theta}{\partial t}+{\color{blue}\mathbf{u}\cdot\nabla\theta}=Q$\tabularnewline
 & \tabularnewline
\multicolumn{2}{l}{An equation of state, eg perfect gas law, $p=\rho RT$}\tabularnewline
\end{tabular}}{\global\long\def\arraystretch{1}

\begin{flushleft}
\begin{tabular}{llll}
$\mathbf{u}$  & Wind vector  & $\vec{g}$  & Gravity vector (downwards)\tabularnewline
$t$  & Time  & $\theta$  & Potential temperature, $T\left(p_{0}/p\right)^{\kappa}$\tabularnewline
$\bm{\Omega}$  & Rotation rate of planet\hspace{1cm} & $\kappa$ & heat capacity ratio $\approx1.4$\tabularnewline
$\rho$  & Density of air & $Q$ & Source of heat\tabularnewline
$p$  & Atmospheric pressure &  & \tabularnewline
\end{tabular}
\par\end{flushleft}

}

\clearpage{}

\section*{Who uses what for Advection?}

\begin{tabular}{lll}
\textbf{Model/Modelling centre} & \textbf{Version} & \textbf{Numerical method}\tabularnewline
\hline 
UK Met Office & Current & \textcolor{blue}{Semi-Lagrangian (not conservative)}\tabularnewline
 & Next & \textcolor{red}{Flux form semi-Lagrangian? (FV)}\tabularnewline
\hline 
ECMWF & Current & \textcolor{blue}{Semi-Lagrangian (not conservative)}\tabularnewline
 & Next & \textcolor{red}{Flux form semi-Lagrangian (FV)}\tabularnewline
\hline 
Environment Canada &  & \textcolor{blue}{Semi-Lagrangian (not conservative)}\tabularnewline
\hline 
NOAA &  & \textcolor{red}{Flux form semi-Lagrangian (FV)}\tabularnewline
\hline 
NCAR CAM & FV & \textcolor{red}{Flux form semi-Lagrangian (FV)}\tabularnewline
 & SE & \textcolor{cyan}{Spectral Element (no upwinding)}\tabularnewline
\hline 
FV3 (NOAA and GFDL) &  & \textcolor{red}{Flux form semi-Lagrangian (FV)}\tabularnewline
\hline 
NUMA-Neptune (US Navy) & Next & \textcolor{violet}{Discontinuous Galerkin}\tabularnewline
\hline 
NICAM (A Japanese model) &  & \textcolor{red}{Flux form semi-Lagrangian (FV)}\tabularnewline
\hline 
MPAS (to replace WRF) &  & \textcolor{olive}{Upwinded finite volume}\tabularnewline
\hline 
ICON &  & \textcolor{red}{Flux form semi-Lagrangian (FV)}\tabularnewline
\hline 
DYNAMICO &  & \textcolor{red}{Flux form semi-Lagrangian (FV)}\tabularnewline
\hline 
CSU &  & \textcolor{olive}{Upwinded finite volume}\tabularnewline
\end{tabular}

\clearpage{}

\section*{Semi-Lagrangian}

\begin{minipage}[t]{0.6\columnwidth}%
\[
\frac{\partial\psi}{\partial t}+\mathbf{u}\cdot\nabla\psi=0
\]
solved as
\[
\psi_{ij}^{n+1}=\psi_{d}^{n}
\]
where $n$ is the time level, $ij$ is the position on the grid and
$d$ is the departure point. 

Interpolate to find $\psi_{d}$ at the departure point from surrounding
values on the grid.%
\end{minipage} %
\begin{minipage}[t]{0.38\columnwidth}%
\phantom{}\resizebox{1\textwidth}{!}{\input{figs/semiLagrange.pdftex_t}}%
\end{minipage}

\subsubsection*{Advantages and Disadvantages}
\begin{description}
\item [{\smiley{}}] \optBlock{Stable and accurate for very long time
steps}
\item [{\smiley{}}] \optBlock{Cost and accuracy not strongly related
to time step}
\item [{\frownie{}}] \optBlock{$\psi$ is not conserved}
\end{description}
\clearpage{}

\begin{minipage}[t]{0.6\columnwidth}%

\section*{Flux Form Semi-Lagrangian}

\textbf{Also known as:}
\begin{itemize}
\item Forward in time
\item Swept area/volume
\item Space-time
\end{itemize}
\textbf{Examples: }
\begin{itemize}
\item PPM (piecewise parabolic method)
\item Lin and Rood
\item COSMIC
\item Lax-Wendroff. \nocite{LLM96,LLM95,CW84,LR96}
\end{itemize}
%
\end{minipage} %
\begin{minipage}[t]{0.38\columnwidth}%
\phantom{}\resizebox{1\textwidth}{!}{\input{figs/hexFlux.pdftex_t}}
\begin{description}
\item [{\smiley{}}] \optBlock{Conservative}
\item [{\frownie{}}] \optBlock{Only used with small time-steps ( $c<1$)}
\end{description}
%
\end{minipage}

\textcolor{purple}{}%
\begin{minipage}[t]{0.33\columnwidth}%
\begin{center}
\textcolor{purple}{
\[
\frac{\partial\psi}{\partial t}+\nabla\cdot\psi\mathbf{u}=0
\]
}
\par\end{center}%
\end{minipage}\textcolor{purple}{}%
\begin{minipage}[t]{0.2\columnwidth}%
\begin{center}
\textcolor{purple}{
\[
\text{solved as }
\]
}
\par\end{center}%
\end{minipage}\textcolor{purple}{}%
\begin{minipage}[t]{0.4\columnwidth}%
\begin{center}
\textcolor{purple}{
\[
\psi^{n+1}=\psi^{n}-\frac{1}{V}\sum_{\text{faces}}\psi_{f}\mathbf{u}\cdot\mathbf{S}
\]
}
\par\end{center}%
\end{minipage}\\
where $\psi_{f}$ is integrated over the volume swept through the
face during one time step. 

\clearpage{}

\section*{Other Advection Schemes}

Space and time discretised separately \textendash{} ``method of lines''. 

\begin{minipage}[t]{0.48\columnwidth}%

\subsubsection*{Options for Space}
\begin{itemize}
\item Finite difference (with upwinding)
\item Finite volume (with upwinding)
\item Spectral Element (not upwinded)
\item Discontinuous Galerkin
\item Finite Element (Petrov-Galerkin with upwinding)
\end{itemize}
%
\end{minipage} %
\begin{minipage}[t]{0.48\columnwidth}%

\subsubsection*{Options for Time}
\begin{itemize}
\item Runge-Kutta (multi-stage)
\item Multi-step (eg leapfrog)
\item Implicit (eg Crank-Nicolson)
\end{itemize}
%
\end{minipage}

\clearpage{}

\section*{Numerical Methods for Gravity and Acoustic waves (2nd order wave
equations)}
\begin{itemize}
\item Finite Difference - Arakawa A, B, C, D or E grids \citep{AL77}
\begin{itemize}
\item Finite Volume (A-grid \textendash{} co-located or C-grid \textendash{}
staggered)
\end{itemize}
\item Spectral element (co-located)
\item Discontinuous Galerkin (co-located)
\item Spectral method (co-located)
\item Finite Element
\begin{itemize}
\item Mixed finite element \textendash{} finite element version of staggered
\end{itemize}
\end{itemize}
\clearpage{}

\begin{minipage}[t]{0.68\columnwidth}%

\section*{Spectral Method}
\begin{itemize}
\item Fourier and Legendre transforms convert grid point data, $\tilde{X}(\lambda,\theta)$
($\lambda$ is longitude and $\theta$ is latitude) into coefficients,
$X_{n}^{m}$ so that the atmospheric state can be represented as a
sum of spherical harmonics:
\[
\tilde{X}(\lambda,\theta)=\sum_{n=0}^{N}\sum_{m=-n}^{n}X_{n}^{m}P_{n}^{m}(\theta)e^{im\lambda}
\]
where $P_{n}^{m}$ is a Legendre polynomial.
\item Differentiation and interpolation can then be done very accurately
in spectral space.
\end{itemize}
\begin{flushright}
\vspace{2cm}
Spherical harmonics for $n=5$ $\rightarrow$
\par\end{flushright}%
\end{minipage} %
\begin{minipage}[t]{0.3\columnwidth}%
\phantom{}\includegraphics[width=1\textwidth]{graphics/Spherical_harmonics}%
\end{minipage}

\clearpage{}

\begin{minipage}[t]{0.55\columnwidth}%

\section*{Spectral Method}
\begin{description}
\item [{\smiley}] \optBlock{Highest possible order of accuracy for the
resolution}
\item [{\smiley}] \optBlock{ECMWF IFS has always been the most accurate
model. }
\item [{\smiley}] \optBlock{No grid imprinting due to gird irregularities}
\item [{\frownie}] \optBlock{Lot of  communication involved in spectral
transforms and inverse transforms $\rightarrow$ parallel scaling
problems}
\item [{\frownie}] \optBlock{Spectral ringing}
\end{description}
%
\end{minipage}\hfill{}%
\begin{minipage}[t]{0.4\columnwidth}%
\begin{center}
\textbf{Scaling of spectral transform}\\
(Andreas M\"ueller, ECMWF)\\
\includegraphics[width=1\textwidth]{graphics/AndreasEscape}
\par\end{center}
\begin{center}
\textbf{Performance modelling}\\
{\small{}\citep{ZM18}}\\
\includegraphics[width=0.9\textwidth]{graphics/scalingDwarfs}
\par\end{center}%
\end{minipage}

\clearpage{}

\begin{minipage}[c][1\totalheight][t]{0.38\columnwidth}%

\subsection*{Finite Element Method}

Representation of a function using a piecewise linear basis.
\begin{center}
\includegraphics[width=1\textwidth]{graphics/Piecewise_linear_function2D}
\par\end{center}%
\end{minipage}\hfill{}%
\begin{minipage}[c][1\totalheight][t]{0.58\columnwidth}%
The solution, $u(\mathbf{x})$, is represented as a sum of $N$ basis
functions:
\begin{equation}
u(\mathbf{x})=\sum_{i=1}^{N}U_{i}\phi_{i}(\mathbf{x}).\label{eq:asBasis}
\end{equation}
For the finite element method, the basis functions, $\phi_{i}$, are
piecewise polynomials defined as non-zero on on each element. 
\begin{itemize}
\item To find $U_{i}$, multiply eqn (\ref{eq:asBasis}) by a test function.
\item Galerkin method:
\begin{itemize}
\item test functions = basis functions
\end{itemize}
\item Integrate by parts to get weak formulation
\item Leads to a set of linear simultaneous equations: the mass matrix (or
stiffness matrix) \textendash{} \textbf{a global matrix}
\item Solve to find $U_{i}$
\end{itemize}
%
\end{minipage}

\clearpage{}

\section*{Spectral Element Method }

\begin{minipage}[t]{0.48\columnwidth}%
Basis functions are Lagrange interpolation on Gauss-Lobatto quadrature
points

\includegraphics[width=1\textwidth]{graphics/GLpoints}%
\end{minipage}\hfill{}%
\begin{minipage}[t]{0.48\columnwidth}%
\begin{itemize}
\item Integration over an element is summation of values at Gauss-Lobatto
quadrature points
\item Leads to a diagonal mass matrix\\
$\therefore$ less communication 
\item Used by CAM-SE on the cubed sphere (4th order accurate)
\item Excellent parallel scaling
\item Pressure and velocity co-located\\
$\therefore$ A-grid computational mode but high order
\end{itemize}
%
\end{minipage}

\clearpage{}

\section*{Mixed Finite Element}
\begin{itemize}
\item Different basis functions for pressure and velocity\\
$\therefore$ good wave dispersion
\item Hear Nigel Wood's talk on next Met Office model
\end{itemize}
\clearpage{}

\section*{Vertical Discretisation}

\vspace{-1cm}
\begin{minipage}[t]{0.66\columnwidth}%
\phantom{}\includegraphics[width=1\textwidth]{graphics/verticalStaggering}%
\end{minipage} %
\begin{minipage}[t]{0.25\columnwidth}%
\begin{center}
\phantom{}\vspace{-0.3cm}
(c) co-located
\par\end{center}
\resizebox{1\textwidth}{!}{\input{figs/vertCoGrid.pdftex_t}}%
\end{minipage}

\ifthenelse{\boolean{@lecversion} \OR \boolean{@onlineversionWith}}{{\color{red}%

\begin{minipage}[t]{0.33\columnwidth}%
\begin{center}
UK Met Office\\
Environment Canada
\par\end{center}%
\end{minipage}%
\begin{minipage}[t]{0.33\columnwidth}%
\begin{center}
MPAS, DYNAMICO\\
ICON, NICAM, CAM-SE
\par\end{center}%
\end{minipage}%
\begin{minipage}[t]{0.33\columnwidth}%
\begin{center}
ECMWF IFS and FVM\\
GFDL and NOAA's FV$^{3}$\\
NUMA-Neptune, CAM-FV?
\par\end{center}%
\end{minipage}

}}{

Which model has which vertical staggering?

UK Met Office\hspace{1cm} Environment Canada\hspace{1cm} MPAS\hspace{1cm}
DYNAMICO\hspace{1cm} ICON\hspace{1cm} NICAM\hspace{1cm} CAM-SE\hspace{1cm}
ECMWF IFS\hspace{1cm} ECMWF FVM\hspace{1cm} GFDL and NOAA's FV$^{3}$\hspace{1cm}
NUMA-Neptune\hspace{1cm} CAM-FV

}

\clearpage{}\nocite{UJK+17}

\nocite{NRC+10}

\nocite{SG11}

\nocite{SKD+12}

\nocite{SMT+08}

\nocite{ZRR+15}

\nocite{DDT15}

\nocite{ECMWF_IFS}

\nocite{comsol_fem}
