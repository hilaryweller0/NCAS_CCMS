
\chapter{Numerical Analysis}

\stepwise{

Numerical analysis is about predicting mathematically how a numerical
method will behave before coding it up and testing it. There are many
aspects of numerical analysis but in this chapter, order of accuracy
and stability will be discussed. 

\step{

\section{Accuracy and Order of Accuracy}

The accuracy of a numerical method is closely associated with the
order of accuracy. If a finite difference scheme uses a grid spacing
$\Delta x$, and if errors are bounded by $C\Delta x^{n}$ for some
constant $C$, then the method is $n$th order accurate. Errors converge
to zero at a rate proportional to $\Delta x^{n}$ as $\Delta x$ decreases.
This can be written $O(\Delta x^{n})$. Accuracy and order of accuracy
can be analysed using Taylor series. }\step{

\subsection{Accuracy of CTCS discretisations}

CTCS uses centred in space and time discretisations for gradients:
\begin{equation}
\frac{\partial\phi}{\partial x}_{j}=\phi_{j}^{\prime}\approx\opttext{\frac{\phi_{j+1}-\phi_{j-1}}{2\Delta x}}
\end{equation}
}\step{To derive this formula and find is order of accuracy, we need
to write down the Taylor series approximations for $\phi_{j-1}$ and
$\phi_{j+1}$ about $\phi_{j}$:
\begin{align*}
\phi_{j-1} & \approx\opttext{\phi_{j}-\Delta x\phi_{j}^{\prime}+\frac{\Delta x^{2}}{2!}\phi_{j}^{\prime\prime}-\frac{\Delta x^{3}}{3!}\phi_{j}^{\prime\prime\prime}+\cdots}\\
\phi_{j+1} & \approx\opttext{\phi_{j}+\Delta x\phi_{j}^{\prime}+\frac{\Delta x^{2}}{2!}\phi_{j}^{\prime\prime}+\frac{\Delta x^{3}}{3!}\phi_{j}^{\prime\prime\prime}+\cdots}\ \ \ \ \ \ \ \ \ \ \ \ \ \ \ \ \ \ \ \ \ \ \ \ \ \ \ \ \ \ \ \ \ \ \ \ \ \ \ \ \ \ \ \ \ \ \ \ \ \ \ .
\end{align*}

}}\clearpage{}\stepwise{

We can assume that we know $\phi_{j}$, $\phi_{j-1}$, $\phi_{j+1}$.
We want to find the most accurate approximation for $\phi_{j}^{\prime}$
so we eliminate the largest unknown, $\Delta x^{2}\phi_{j}^{\prime\prime}$,
and rearrange to find $\phi_{j}^{\prime}$:

\begin{align*}
\implies\phi_{j+1}-\phi_{j-1} & \approx\opttext{2\Delta x\phi_{j}^{\prime}+\frac{\Delta x^{3}}{3}\phi_{j}^{\prime\prime\prime}}\ \ \ \ \ \ \ \ \ \ \ \ \ \ \ \ \ \ \ \ \ \ \ \ \ \ \ \ \ \ \ \ \ \ \ \ \ \ \ \ \ \ \ \ \ \ \ \ \ \ \ .\\
\implies\phi_{j}^{\prime} & \approx\opttext{\frac{\phi_{j+1}-\phi_{j-1}}{2\Delta x}-\frac{\Delta x^{2}}{3!}\phi_{j}^{\prime\prime\prime}}
\end{align*}
\vspace{0.1cm}
\step{
\begin{itemize}
\item $\phi_{j}^{\prime\prime\prime}$ is unknown so the leading error term
is $\frac{\Delta x^{2}}{3!}\phi_{j}^{\prime\prime\prime}$ so CTCS
is second order accurate.\step{
\item The error is proportional to $\phi^{\prime\prime\prime}=\partial^{3}\phi/\partial x^{3}$.
This term makes the numerical solution dispersive - this is a term
in a dispersion equation.}
\end{itemize}
}}

\clearpage{}

\subsection{Exercises}
\begin{enumerate}
\item Use the Taylor series to find an approximation for $f_{j}^{\prime}$
in terms of $f_{j}$ and $f_{j-1}$. What order accuracy is it?\vfill{}
\item Derive an uncentred, second order difference formula for $f_{j}^{\prime}$
that uses $f_{j}$, $f_{j+1}$ and $f_{j+2}$. (And show that it is
second order accurate) \vfill{}
\item Find a uncentred approximation for $f_{j}^{\prime\prime}$ using $f_{j}$,
$f_{j+1}$ and $f_{j+2}$. What order accurate is it? 
\item Derive a second order approximation for $f_{b}^{\prime}$ from $f_{a}$,
$f_{b}$ and $f_{c}$ at $x$ locations $a<b<c$ when the grid spacing
is not regular. (And show that it is second order accurate) 
\end{enumerate}
\clearpage{}

\optpage{

\subsubsection*{Answers}
\begin{enumerate}
\item Use the Taylor series to find an approximation for $f_{j}^{\prime}$
in terms of $f_{j}$ and $f_{j-1}$. What order accuracy is it?

Write the Taylor series for $f_{j-1}$ in terms of $f_{j}$:\\
$f_{j-1}=f_{j}-\Delta xf_{j}^{\prime}+O(\Delta x^{2})$\\
Rearrange to find $f_{j}^{\prime}$:\\
$f_{j}^{\prime}=(f_{j}-f_{j-1})/\Delta x+O(\Delta x)$ \\
Note dividing $O(\Delta x^{2})$ by $\Delta x$ gives $O(\Delta x)$
so the approximation is first order accurate
\item Derive an uncentred, second order difference formula for $f_{j}^{\prime}$
that uses $f_{j}$, $f_{j+1}$ and $f_{j+2}$. (And show that it is
second order accurate)

Taylor approximations for $f_{j+1}$ and $f_{j+2}$ about $f_{j}$:\\
$f_{j+1}=f_{j}+\Delta xf_{j}^{\prime}+\frac{\Delta x^{2}}{2!}f_{j}^{\prime\prime}+\frac{\Delta x^{3}}{3!}f_{j}^{\prime\prime\prime}+\frac{\Delta x^{4}}{4!}f_{j}^{\prime\prime\prime\prime}+O(\Delta x^{5})$\\
$f_{j+2}=f_{j}+2\Delta xf_{j}^{\prime}+2\Delta x^{2}f_{j}^{\prime\prime}+\frac{4\Delta x^{3}}{3}f_{j}^{\prime\prime\prime}+\frac{2\Delta x^{4}}{3}f_{j}^{\prime\prime\prime\prime}+O(\Delta x^{5})$\\
Eliminate the largest unknown, $f_{j}^{\prime\prime}$ by calculating
$f_{j+2}-4f_{j+1}$:\\
$f_{j+2}-4f_{j+1}=-3f_{j}-2\Delta x\ f_{j}^{\prime}+O(\Delta x^{3})$\\
Rearrange to find $f_{j}^{\prime}$:\\
$f_{j}^{\prime}=\left(-f_{j+2}+4f_{j+1}-3f_{j}\right)/(2\Delta x)+O(\Delta x^{2})$
\item Find an uncentred approximation for $f_{j}^{\prime\prime}$ using
$f_{j}$, $f_{j+1}$ and $f_{j+2}$. What order accurate is it? 

Taylor approximations for $f_{j+1}$ and $f_{j+2}$ about $f_{j}$:\\
$f_{j+1}=f_{j}+\Delta xf_{j}^{\prime}+\frac{\Delta x^{2}}{2!}f_{j}^{\prime\prime}+\frac{\Delta x^{3}}{3!}f_{j}^{\prime\prime\prime}+\frac{\Delta x^{4}}{4!}f_{j}^{\prime\prime\prime\prime}+O(\Delta x^{5})$\\
$f_{j+2}=f_{j}+2\Delta xf_{j}^{\prime}+2\Delta x^{2}f_{j}^{\prime\prime}+\frac{4\Delta x^{3}}{3}f_{j}^{\prime\prime\prime}+\frac{2\Delta x^{4}}{3}f_{j}^{\prime\prime\prime\prime}+O(\Delta x^{5})$\\
Eliminate the largest unknown, $f_{j}^{\prime}$ by calculating $f_{j+2}-2f_{j+1}$:\\
$f_{j+2}-2f_{j+1}=-f_{j}-\Delta x^{2}\ f_{j}^{\prime\prime}+O(\Delta x^{3})$\\
Rearrange to find $f_{j}^{\prime\prime}$:\\
$f_{j}^{\prime\prime}=\left(-f_{j+2}+2f_{j+1}-f_{j}\right)/\Delta x^{2}+O(\Delta x)$
\item Derive a second order approximation for $f_{b}^{\prime}$ from $f_{a}$,
$f_{b}$ and $f_{c}$ at $x$ locations $a<b<c$ when the grid spacing
is not regular. (And show that it is second order accurate)

Define $\Delta x_{1}=b-a$ and $\Delta x_{2}=c-b$ and $\Delta x=\max\left(\Delta x_{1},\Delta x_{2}\right)$\\
Taylor approximations for $f_{a}$ and $f_{c}$ about $f_{b}$:\\
$f_{a}=f_{b}-\Delta x_{1}\ f_{b}^{\prime}+\frac{\Delta x_{1}^{2}}{2!}f_{b}^{\prime\prime}-\frac{\Delta x_{1}^{3}}{3!}f_{b}^{\prime\prime\prime}+\frac{\Delta x_{1}^{4}}{4!}f_{b}^{\prime\prime\prime\prime}+O(\Delta x_{1}^{5})$
\\
$f_{c}=f_{b}+\Delta x_{2}\ f_{b}^{\prime}+\frac{\Delta x_{2}^{2}}{2!}f_{b}^{\prime\prime}+\frac{\Delta x_{2}^{3}}{3!}f_{b}^{\prime\prime\prime}+\frac{\Delta x_{2}^{4}}{4!}f_{b}^{\prime\prime\prime\prime}+O(\Delta x_{2}^{5})$\\
Eliminate the largest unknown, $f_{b}^{\prime\prime}$:\\
$\Delta x_{1}^{2}f_{c}-\Delta x_{2}^{2}f_{a}=\Delta x_{1}^{2}\left\{ f_{b}+\Delta x_{2}f_{b}^{\prime}+O(\Delta x_{2}^{3})\right\} -\Delta x_{2}^{2}\left\{ f_{b}-\Delta x_{1}f_{b}^{\prime}+O(\Delta x_{1}^{3})\right\} $\\
Rearrange to find $f_{b}^{\prime}$:\\
$\Delta x_{1}^{2}f_{c}-\Delta x_{2}^{2}f_{a}=\Delta x_{1}^{2}f_{b}+\Delta x_{1}^{2}\Delta x_{2}f_{b}^{\prime}+O(\Delta x_{2}^{3}\Delta x_{1}^{2})-\Delta x_{2}^{2}f_{b}+\Delta x_{1}\Delta x_{2}^{2}f_{b}^{\prime}+O(\Delta x_{1}^{3}\Delta x_{2}^{2})$\\
$\implies\Delta x_{1}^{2}f_{c}-\Delta x_{2}^{2}f_{a}=\left(\Delta x_{1}^{2}-\Delta x_{2}^{2}\right)f_{b}+\Delta x_{1}\Delta x_{2}\left(\Delta x_{1}+\Delta x_{2}\right)f_{b}^{\prime}+O(\Delta x^{5})$\\
$\implies f_{b}^{\prime}=\frac{\Delta x_{1}^{2}f_{c}-\Delta x_{2}^{2}f_{a}-\left(\Delta x_{1}^{2}-\Delta x_{2}^{2}\right)f_{b}}{\Delta x_{1}\Delta x_{2}\left(\Delta x_{1}+\Delta x_{2}\right)}+O(\Delta x^{2})$
\end{enumerate}
}

\clearpage{}

\section{Stability of Numerical Methods}

\subsection{Ordinary Differential Equations (ODEs)}

This section will consider the stability of numerical methods for
systems of linear ordinary differential equations (ODEs, with time-stepping
but no spatial differentials). 

\stepwise{\step{

We are numerically solving an equation of the form
\begin{equation}
\frac{dy}{dt}=\lambda y.\label{eq:1dlinODE}
\end{equation}
where $y$ and $\lambda$ could by complex, $y$ could be a vector
and $\lambda$ could be a matrix. }\step{To predict stability of
the numerical method, we find an amplification factor, $A$, such
that 
\begin{equation}
y^{(n+1)}=Ay^{(n)}
\end{equation}
where $y^{(n)}$ and $y^{(n+1)}$ are the values of $y$ predicted
by the numerical method at time-steps $n$ and $n+1$.}\step{
\begin{itemize}
\item If $A$ is a real number, then the numerical method is stable $\iff|A|\le1$\step{
\item If $A$ is a complex number then the numerical method is stable $\iff||A||\le1$}\step{
\item If $A$ is a matrix then the numerical method is stable if $\rho(A)\le1$
where $\rho(A)$ is the spectral radius of $A$; the largest eigenvalue
of $A$.}
\end{itemize}
}}

\clearpage{}

\subsection{Example}

\stepwise{Consider the explicit, first-order Euler method for solving
eqn (\ref{eq:1dlinODE}):
\begin{equation}
y^{(n+1)}=y^{(n)}+\Delta t\lambda y^{(n)}.
\end{equation}
\step{Substitute in $y^{(n+1)}=Ay^{(n)}$ and divide through by $y^{(n)}$:
\begin{equation}
A=\opttext{1-\lambda\Delta t}.
\end{equation}
}\step{Assuming that $\lambda$ is real, we require $|A|\le1$ for
stability which occurs when:
\begin{alignat*}{3}
 & -1 & \le & 1-\lambda\Delta t & \le & 1\\
\iff & 0 & \le & -\lambda\Delta t & \le & 2\\
\iff & \lambda & \le & 0\\
\text{and }\  & \Delta t & \le & -2/\lambda
\end{alignat*}
}}\clearpage{}

\subsubsection{Exercises\label{ex:ODEstab}}
\begin{enumerate}
\item Find the stability characteristics of solving the oscillation equation
\begin{equation}
\frac{dy}{dt}=i\lambda y\label{eq:oscEqn}
\end{equation}
using Euler forward time-stepping.
\item Find the stability characteristics of solving the oscillation equation
(\ref{eq:oscEqn}) using Euler backward time-stepping:
\begin{equation}
y^{(n+1)}=y^{(n)}+i\Delta t\lambda y^{(n+1)}.
\end{equation}
\item The off-centred Crank-Nicholson scheme for solving (\ref{eq:oscEqn})
is:
\begin{equation}
\frac{y^{(n+1)}-y^{(n)}}{\Delta t}=i\lambda\left(\alpha y^{(n+1)}+(1-\alpha)y^{(n)}\right)
\end{equation}
with off-centering defined by $\alpha$. Find the values of $\alpha$
for which this scheme is stable. Also find the value of $\alpha$
for which the scheme is not damping (a scheme is damping if $||A||<1$).
\end{enumerate}
\optpage{

\subsubsection*{Answers}
\begin{enumerate}
\item $A=1-i\lambda\Delta t\ \implies\ ||A||^{2}=1+\lambda^{2}\Delta t^{2}\ \implies$
the scheme is unconditionally unstable \textendash{} unstable for
all values of $\Delta t>0$.
\item $A=1/(1-i\Delta t\lambda)\ \implies\ ||A||^{2}=1/(1+\lambda^{2}\Delta t^{2})\ \implies$
the scheme is unconditionally stable \textendash{} stable for all
values of $\Delta t>0$.
\item $A=\left(1+i(1-\alpha)\lambda\Delta t\right)/\left(1+i\alpha\lambda\Delta t\right)$\\
$\implies||A||^{2}=\left(1+(1-\alpha)^{2}\lambda^{2}\Delta t^{2}\right)/\left(1+\alpha^{2}\lambda^{2}\Delta t^{2}\right)$\\
$\therefore\ ||A||^{2}\le1\ \iff\ \alpha^{2}\lambda^{2}\Delta t^{2}\ge(1-\alpha)^{2}\lambda^{2}\Delta t^{2}$\\
$\iff\alpha\ge1-\alpha\ \iff\ \alpha\ge1/2$.
\end{enumerate}
}

\clearpage{}

\subsection{Von-Neumann Stability Analysis}

\stepwise{

The initial conditions of any PDE, or the solution at any time, can
be written as a sum of Fourier modes:
\begin{equation}
\phi(x,t)=\sum_{k}A_{k}(t)\ e^{ikx}
\end{equation}
Linear numerical methods, operating on linear PDEs, operate independently
on each Fourier mode of a solution. Therefore we only need to analyse
the influence of a numerical method on a Fourier mode. 

\step{We will consider the one-dimensional linear advection equation:
\begin{equation}
{\displaystyle \frac{\partial\phi}{\partial t}+u\frac{\partial\phi}{\partial x}=0}
\end{equation}

}\step{and the forward in time, backward in space (FTBS) scheme for
linear advection:
\begin{equation}
\frac{\phi_{j}^{(n+1)}-\phi_{j}^{(n)}}{\Delta t}+u\frac{\phi_{j}^{(n)}-\phi_{j-1}^{(n)}}{\Delta x}=0
\end{equation}
}\step{which can be re-arranged to give $\phi_{j}^{(n+1)}$ in terms
of the $\phi_{j}$s at the previous time-step and the Courant number,
$c=u\Delta t/\Delta x$}\step{: 
\begin{equation}
\phi_{j}^{(n+1)}=\phi_{j}^{(n)}-c\bigl(\phi_{j}^{(n)}-\phi_{j-1}^{(n)}\bigr)\label{eq:FTBS}
\end{equation}
}\step{Replace the $\phi$s in eqn (\ref{eq:FTBS}) with one Fourier
mode and an amplification factor that multiplies the solution at every
time-step:
\begin{equation}
\phi_{j}^{(n)}=A^{n}e^{ikj\Delta x}
\end{equation}
giving:
\begin{equation}
\opttext{A^{n+1}e^{ikj\Delta x}=A^{n}e^{ikj\Delta x}-c\left(A^{n}e^{ikj\Delta x}-A^{n}e^{ik(j-1)\Delta x}\right)}
\end{equation}

}}

\clearpage{}\stepwise{Cancelling powers of $A^{n}e^{ikj\Delta x}$
gives an expression for $A$:

\begin{equation}
A=\opttext{1-c(1-e^{-ik\Delta x})}
\end{equation}
\step{We need to find the magnitude of $A$ so we need to write it
down in real and imaginary form. So substitute $e^{-ik\Delta x}=\cos k\Delta x-i\sin k\Delta x$:
\begin{equation}
A=\opttext{1-c(1-\cos k\Delta x)-ic\sin k\Delta x}
\end{equation}
}\step{and calculate $|A|^{2}=AA^{*}$ ($A$ multiplied by its complex
conjugate):
\begin{align*}
|A|^{2} & =1-2c(1-\cos k\Delta x)+c^{2}(1-2\cos k\Delta x+\cos^{2}k\Delta x)+c^{2}\sin^{2}k\Delta x\\
\implies|A|^{2} & =1-2c(1-c)(1-\cos k\Delta x)
\end{align*}
}\step{We need to find for what value of $\Delta t$ or $c$ is $|A|\le1$
in order to find when FTBS is stable: 
\begin{align*}
|A|\le1 & \iff|A|^{2}-1\le0\\
 & \iff-2c(1-c)(1-\cos k\Delta x)\le0\\
 & \iff c(1-c)(1-\cos k\Delta x)\ge0
\end{align*}

}\step{We know that $1-\cos k\Delta x\ge0$ for all $k\Delta x$
so FTBS is stable when\\
 $c(1-c)\ge0\iff0\le c\le1$.

We have proved that FTBS is unstable if $u<0$ or if $\frac{u\Delta t}{\Delta x}>1$.
We will now define:

\begin{tabular}{lll}
\textbf{Upwind scheme } & FTBS  & when $u\ge0$\tabularnewline
 & FTFS  & when $u<0$ \tabularnewline
\end{tabular}

The upwind scheme is first order accurate in space and time, conditionally
stable and damping.

}}

\clearpage{}

\subsubsection*{Exercises}
\begin{enumerate}
\item Find the amplification factor and hence determine the stability of
the forward in time, centred in space (FTCS) scheme for linear advection:
\[
\phi_{j}^{(n+1)}=\phi_{j}^{(n)}-\frac{c}{2}\left(\phi_{j+1}^{(n)}-\phi_{j-1}^{(n)}\right)
\]
\item Find the amplification factor and hence determine the stability of
the forward in time, centred in space (FTCS) scheme for the heat equation:
\[
\frac{T_{j}^{(n+1)}-T_{j}^{(n)}}{\Delta t}=\frac{K}{\Delta x^{2}}\left(T_{j-1}^{(n)}-2T_{j}^{(n)}+T_{j+1}^{(n)}\right).
\]
\end{enumerate}
\clearpage{}

\optpage{

\subsubsection*{Solutions}
\begin{enumerate}
\item FTCS: $\phi_{j}^{(n+1)}=\phi_{j}^{(n)}-\frac{c}{2}\left(\phi_{j+1}^{(n)}-\phi_{j-1}^{(n)}\right)$\\
Substitute in $\phi_{j}^{(n)}=A^{n}e^{ikj\Delta x}$ and cancel powers
of $A^{n}e^{ikj\Delta x}$:
\begin{align*}
A= & 1-\frac{c}{2}(e^{ik\Delta x}-e^{-ik\Delta x})\\
= & 1-ic\sin k\Delta x
\end{align*}
Find for what Courant numbers the scheme is stable: 
\begin{align*}
A= & 1-ic\sin k\Delta x\\
\Rightarrow|A|^{2}= & 1+c^{2}\sin^{2}k\Delta x
\end{align*}
which is $>1$ for all $|c|>0$ and so FTCS is unconditioanlly unstable
\item For the FTCS scheme for diffusion we use the shorthand $d=\frac{K\Delta t}{\Delta x^{2}}$.
Then:
\begin{eqnarray}
A & = & 1+d(e^{ik\Delta x}-2+e^{-ik\Delta x})\\
 & = & 1-2d(1-\cos k\Delta x)
\end{eqnarray}
We require the scheme to be stable for all wavenumbers, $k$. So find
for what values of $d$ the scheme is stable for all values of $k\Delta x$.
\begin{align*}
 & -1\le A\le1\\
\iff & -1\le1-2d(1-\cos k\Delta x)\le1\\
\iff & 1\ge-1+2d(1-\cos k\Delta x)\ge-1\ \text{ add one to all 3 sides}\\
\iff & 2\ge2d(1-\cos k\Delta x)\ge0\ \text{ divide through by }2d\ \text{(which is always positive) }\\
\iff & 0\le(1-\cos k\Delta x)\le\frac{1}{d}
\end{align*}
Sketch $1-\cos k\Delta x$ to find its maximum and minimum values.
From this we can see that $0\le(1-\cos k\Delta x)\le2$ for any value
of $k\Delta x$. For stability we need $(1-\cos k\Delta x)\le\frac{1}{d}$
for \textit{every} value of $k\Delta x$ so we need:
\[
2\le\frac{1}{d}\ \iff d\le\frac{1}{2}\ \iff K\Delta t/\Delta x^{2}\le\frac{1}{2}\ \iff\Delta t\le\frac{\Delta x^{2}}{2K}
\]
If in addition we do not want the solution to oscillate in sign from
one time-step (ie non-oscillatory) to the next then we require: 
\begin{align*}
 & 0\le A\le1\\
\iff & 0\le1-2d(1-\cos k\Delta x)\le1\\
\iff & 0\ge-1+2d(1-\cos k\Delta x)\ge-1\\
\iff & 1/2\ge d(1-\cos k\Delta x)\ge0\\
\iff & 1/(2(1-\cos k\Delta x))\ge d\ge0~\text{ since }~0\le1-\cos k\Delta x\le2\\
\Leftarrow~~ & d\le\frac{1}{4}\iff\Delta t\le\frac{\Delta x^{2}}{4K}
\end{align*}
So for large $K$ or for high resolution (small $\Delta x$) a very
small time-step must be taken for FTCS to be stable
\end{enumerate}
}
