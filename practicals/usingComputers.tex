\clearpage{}

\section{Using the Computers for the Practicals}
\begin{enumerate}
\item Once you have logged in to a workstation, open a terminal window.
\item Create and move to a directory for storing everything related to the
notes and practicals for this week:\\
\texttt{mkdir -p climateModelling/numerics}\\
\texttt{cd climateModelling/numerics}
\item Create a test directory to write your first python program:\\
\texttt{mkdir testPython}~\\
\texttt{cd testPython}~\\
\texttt{gedit helloWorld.py \&}\\
(The \texttt{\&} makes the command run in the background). \\
Insert the following python code into your new file:\\
\begin{lstlisting}[language=Python]
import numpy as np
r = 4
print 'Area of circle of radius', r, ' is ', np.pi*r**2
\end{lstlisting}
Then save the file
\item Run this code interactively (using python 2.7):\\
\texttt{python}\\
and in python type:\\
\texttt{execfile(\textquotedbl{}helloWorld.py\textquotedbl{})}~\\
or from the command line:\\
\texttt{python helloWorld.py}
\item To leave python, type \texttt{exit()}
\item \textbf{Very important: tabs versus spaces.} If you mix tabs and spaces
in python, it will behave in unexpected ways. I recommend using 4
spaces instead of tabs. So you should setup your text editor (I recommend
gedit) to put in 4 spaces instead of a tab. In \texttt{gedit, select
``automatic indendation'', 4 space tabs and ``Use spaces'' in
the ``Tab Width'' option at the bottom.}
\item Some essential unix commands:

\begin{tabular}{ll}
\texttt{mkdir dirName}  & Make directory named ``dirName'' \tabularnewline
\texttt{cd dirName}  & Change directory to directory named ``dirName''\tabularnewline
\texttt{cd ..} & Move back one directory\tabularnewline
\texttt{cp file1 file2}  & Copy file named ``file1'' to location ``file2''\tabularnewline
\texttt{cp -r dir1 dir2}  & Copy the entire directory ``dir1'' to location ``dir2''\tabularnewline
\texttt{ls}  & List contents of the current directory \tabularnewline
\texttt{pwd}  & Print working directory \tabularnewline
\texttt{mv file1 file2}  & Move file (or directory) ``file1'' to file or directory named ``file2'' \tabularnewline
\texttt{rm file1}  & Remove (delete) ``file1'' \tabularnewline
\texttt{rm -r dir1}  & Remove (delete) directory structure ``dir1'' \tabularnewline
\texttt{man command}  & Print the manual page for ``command''\tabularnewline
\texttt{more file1}  & Write out the contents of ``file1'' one page at a time\tabularnewline
 & (get another page by pressing space) \tabularnewline
\end{tabular}
\item You can use \texttt{evince} to view pdf files. Eg:

\texttt{evince WellerSSnotes\_2\_lec.pdf \&}
\end{enumerate}

