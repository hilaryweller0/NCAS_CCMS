
\chapter{Solving an Equation using Finite Differences}

This practical is open ended and we will build on it during the week.
Everyone has different knowledge and experience coding numerical methods
so you will choose your own equation and your own numerical method
to solve it. Each section of this chapter describes some alternative
starting points in increasing levels of difficulty. Read through sections
\ref{sec:solveOscEqn} to \ref{sec:solveSWE} before deciding which
to work on. Focus on just one and dive deep. On Thursday, you will
give a short presentation to the class on what equation you have solved,
how you have solved it, your results, what was most difficult and,
most importantly, \textbf{what you have learned}.
\begin{center}
\textcolor{red}{Do not copy and paste code from the notes}
\par\end{center}

\section{The Oscillation Equation \protect\label{sec:solveOscEqn}}

The code for oscillations is that same as the code for a simple pendulum:
\begin{eqnarray*}
\frac{du}{dt}=fv &  & \frac{dv}{dt}=-fu.
\end{eqnarray*}
This represents inertial oscillations when $(u,v)$ are the components
of the wind and $f$ is the Coriolis parameter. To represent a pendulum,
$u$ would be the angle of the pendulum, $fv$ the speed of the pendulum
and $f$ related to the length of the pendulum. Here are some tasks
you can explore:
\begin{enumerate}
\item Type in the code to solve the oscillation equation (section \ref{sec:intertial_FB}).
\begin{enumerate}
\item Experiment with changing the time step. How does this affect accuracy?
\item The forward-backward method is only stable for a sufficiently small
time step; when the time step is too big the solution diverges. The
stability depends on the product $f\Delta t$. Find the value of $f\Delta t$
above which the method is unstable.\clearpage{}
\end{enumerate}
\item Here are some alternative methods for solving the oscillation equation:\\
\begin{minipage}[t]{0.45\columnwidth}%
\begin{center}
Euler Forward
\begin{eqnarray*}
u^{(n+1)} & = & u^{(n)}+\Delta t\ f\ v^{(n)}\\
v^{(n+1)} & = & v^{(n)}+\Delta t\ f\ u^{(n)}
\end{eqnarray*}
\par\end{center}
\begin{center}
Euler Backward
\begin{eqnarray*}
u^{(n+1)} & = & u^{(n)}+\Delta t\ f\ v^{(n+1)}\\
v^{(n+1)} & = & v^{(n)}+\Delta t\ f\ u^{(n+1)}
\end{eqnarray*}
\par\end{center}
\begin{center}
Crank-Nicolson
\begin{eqnarray*}
u^{(n+1)} & = & u^{(n)}+\frac{\Delta t}{2}\ f\ \left(v^{(n)}+v^{(n+1)}\right)\\
v^{(n+1)} & = & v^{(n)}+\frac{\Delta t}{2}\ f\ \left(u^{(n)}+u^{(n+1)}\right)
\end{eqnarray*}
\par\end{center}%
\end{minipage}\hfill{}%
\begin{minipage}[t]{0.45\columnwidth}%
\begin{center}
Runge-Kutta 4
\begin{eqnarray*}
u_{1} & = & u^{(n)}+\Delta t\ f\ v^{(n)}\\
v_{1} & = & v^{(n)}+\Delta t\ f\ u^{(n)}\\
u_{2} & = & u^{(n)}+\frac{\Delta t}{2}\ f\ v_{1}\\
v_{2} & = & v^{(n)}+\frac{\Delta t}{2}\ f\ u_{1}\\
u_{3} & = & u^{(n)}+\frac{\Delta t}{2}\ f\ v_{2}\\
v_{3} & = & v^{(n)}+\frac{\Delta t}{2}\ f\ u_{2}\\
u_{4} & = & u^{(n)}+\Delta t\ f\ v_{3}\\
v_{4} & = & v^{(n)}+\Delta t\ f\ u_{3}\\
u^{(n+1)} & = & u^{(n)}+\frac{\Delta t}{6}\left(u_{1}+2u_{2}+2u_{3}+u_{4}\right)\\
v^{(n+1)} & = & v^{(n)}+\frac{\Delta t}{6}\left(v_{1}+2v_{2}+2v_{3}+v_{4}\right)
\end{eqnarray*}
\par\end{center}%
\end{minipage} \\
Code up some alternative methods and compare their behaviour with
forward-backwards.
\end{enumerate}
\clearpage{}

\section{The Diffusion Equation}

The one-dimensional, linear diffusion equation is given in question
2 of exercises \ref{sec:FD_mathEx}. 
\begin{enumerate}
\item Write code to solve it using the forward in time, centered in space
method. Use periodic boundary conditions, $x\in[0,1)$, $K=1$ and
initial conditions
\[
\phi_{0}=\begin{cases}
1 & x\in\left[0.4,0.6\right]\\
0 & \text{otherwise}.
\end{cases}
\]
\item With these initial conditions, the analytic solution for short timescales
can be approximated by
\[
\phi\left(x,t\right)=\frac{1}{2}\left[\text{erf}\left(\frac{x-0.4}{\sqrt{4Kt}}\right)-\text{erf}\left(\frac{x-0.6}{\sqrt{4Kt}}\right)\right].
\]
Write code to compare your numerical solutions with the analytic solution.
\item Investigate how the solution behaves when you change the spatial and
temporal resolution.
\item The diffusion equation can be used to represent the variation of temperature,
$T$, with height, $z$, in the atmospheric boundary layer:
\[
\frac{\partial T}{\partial t}=K\frac{\partial^{2}T}{\partial z^{2}}+Q
\]
where $Q$ is a heating rate. Write code to solve this equation in
the lowest 1000\,m of the atmosphere with $K=1\ \text{m}^{2}\text{s}^{-1}$,
$Q=-1.5\ \text{K}/\text{day}$, subject to boundary and initial conditions:
\begin{eqnarray*}
\text{Bottom boundary, }T\left(0,t\right) & = & 293\ \text{K}\\
\text{Top boundary (adiabatic), }\frac{\partial T}{\partial z}\left(1000,t\right) & = & 0\\
\text{Initial conditions, }T\left(z,0\right) & = & 293\ \text{K}
\end{eqnarray*}
Think about how to validate your code.\clearpage{}
\end{enumerate}

\section{The linear advection equation}
\begin{enumerate}
\item Type in the code given in section \ref{sec:CTCScode} to solve the
linear advection equation with periodic boundary conditions.
\item Experiment to find how the method behaves when you change the spatial
and temporal resolution.
\item Experiment to find how the method behaves when you use initial conditions
with discontinuities, such as:
\[
\phi_{0}=\begin{cases}
1 & x\in\left[0.1,0.5\right]\\
0 & \text{otherwise}
\end{cases}
\]

When advecting fields with discontinuities, centered in space methods
such as CTCS show severe dispersion errors leading to spurious undershoots
and overshoots. There is a vast literature on advection schemes designed
to limit these dispersion errors. Here is a selection of methods you
could try. For each of these, investigate the advantages and disadvantages:
\begin{enumerate}
\item Add a diffusion term, $K\nabla^{2}\phi$, to the right hand side of
the advection equation.
\item Use forward in time, backward in space, FTBS, instead of CTCS.
\item Use an upwinded, finite volume scheme such as Lax-Wendroff or Warming
and Beam:
\[
\frac{\phi_{j}^{(n+1)}-\phi_{j}^{(n)}}{\Delta t}=-u\frac{\phi_{j+\half}-\phi_{j-\half}}{\Delta x}
\]
Lax-Wendroff: $\phi_{j+\half}=\half(1+c)\phi_{j}^{(n)}+\half(1-c)\phi_{j+1}^{(n)}$\\
Warming and Beam: $\phi_{j+\half}=\half(3-c)\phi_{j}^{(n)}-\half(1-c)\phi_{j-1}^{(n)}$
\item Use a semi-Lagrangian scheme. The semi-Lagrangian method approximates
the analytic solution of the advection equation:
\begin{eqnarray*}
\phi\left(x,t+\Delta t\right) & = & \phi\left(x-u\Delta t,t\right).
\end{eqnarray*}
Point $x_{jd}=x_{j}-u\Delta t$ is the departure point of point $x_{j}$
which, let's say, falls between grid points $k$ and $k+1$. Can you
work out how to calculate $k$? Once you have $k$, you need to work
out how to interpolate $\phi$ from points $x_{k-1}$, $x_{k}$, $x_{k+1}$,
$x_{k+2}$ onto $x_{jd}$ in order to predict $\phi_{j}^{(n+1)}=\phi_{jd}^{(n)}$
(for example using cubic-Lagrange interpolation).
\end{enumerate}
\end{enumerate}
\clearpage{}

\section{The non-linear advection equation (Burger's equation)\protect\label{sec:solveBurger}}

Derive and code up a numerical method to solve Burger's equation.
Burger's equation has two equivalent forms:
\begin{eqnarray*}
\frac{\partial u}{\partial t}+u\frac{\partial u}{\partial x}=0 &  & \frac{\partial u}{\partial t}+\frac{1}{2}\frac{\partial u^{2}}{\partial x}=0
\end{eqnarray*}

\begin{enumerate}
\item Can you create a method that conserves the total sum of $u$ over
the domain?
\item Investigate how your method behaves when you vary the spatial and
temporal resolution.
\item Is there any way you can know what the correct solution is?
\item Try using one or two of the numerical methods described for linear
advection but adapted for Burger's equation.
\end{enumerate}
\clearpage{}

\section{The linear or non-linear shallow-water equations \protect\label{sec:solveSWE}}

The one-dimensional, linear shallow water equations are:
\begin{eqnarray*}
\frac{\partial u}{\partial t}=-g\frac{\partial h}{\partial x} &  & \frac{\partial h}{\partial t}=-H\frac{\partial u}{\partial x}
\end{eqnarray*}
where $u$ is the fluid velocity, $H+h$ is the total fluid height
with $h<<H$ and $H$ is uniform in space and time. Derive or find
and code up a numerical method to solve the linearised shallow water
equations in a periodic domain $x\in[0,1)$ with $H=1$, $g=1$ using
the initial conditions 
\begin{eqnarray*}
u\left(x,0\right) & = & 0\\
h\left(x,0\right) & = & \begin{cases}
\frac{1}{2}\left(1-\cos2\pi\frac{x-0.4}{0.6-0.4}\right) & x\in\left[0.4,0.6\right]\\
0 & \text{otherwise}.
\end{cases}
\end{eqnarray*}

\begin{enumerate}
\item Investigate how your code behaves when you vary the spatial and temporal
resolution. Are there any particularly unrealistic ways in which your
code behaves for certain set ups?
\item Is there any way you can know what the correct solution is?
\item The code that you wrote is probably co-located or Arakawa A-grid.
This means that $u$ and $h$ are stored at the same locations. Write
a code using a staggered, C-grid so that the locations of the points
where $u$ and $h$ are stored are offset by $\Delta x/2$. This means
that you can use compact differences; if $h$ is stored at locations
$j$ and $u$ is stored at locations $j+\frac{1}{2}$, then, for example,
you can calculate $\partial u/\partial x$ and location $j$, where
it is needed to increment $h_{j}$
\begin{eqnarray*}
h_{j}^{(n+1)}=h_{j}^{(n)}-\Delta tH\frac{u_{j+\frac{1}{2}}-u_{j-\frac{1}{2}}}{\Delta x}, &  & u_{j+\frac{1}{2}}^{(n+1)}=u_{j+\frac{1}{2}}^{(n)}-\Delta tg\frac{h_{j+1}-h_{j}}{\Delta x}.
\end{eqnarray*}
How does the behaviour of the C-grid code compare with that of the
A-grid code?
\end{enumerate}

