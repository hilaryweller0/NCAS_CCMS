
\chapter*{More on the Practical}

\addcontentsline{toc}{chapter}{\protect\numberline{}More on the  Practical}

For this practical you have more choices. You can dig deeper into
what you were doing yesterday, or change direction. Discuss your plans
with staff and each other. Some options:
\begin{enumerate}
\item Develop ways to test your code (essential for everyone).
\item Further analysis of the method that you have already chosen (numerical
analysis or numerical experiment, see below).
\item Implement a better numerical method for the equation you are solving,
such as the spectral method (section \ref{sec:spectralMethod}).
\item Solve a different equation.
\item Download some climate data and do some spectral analysis (section
\ref{sec:powerSpectra}). For example, you could remove a seasonal
cycle before calculating the power spectrum, or look into smoothing
the spectrum, or Fourier filtering the data.
\end{enumerate}

\subsection*{In particular, you may consider:}
\begin{enumerate}
\item If you are solving an advection equation, you might like to consider
using one of:
\begin{enumerate}
\item A monotonic advection scheme
\item Semi-Lagrangian advection
\item The Lax-Wendroff method
\item The Spectral method
\end{enumerate}
\item Analyse how solution errors vary with resolution in time and space
(look at order of convergence)?
\item Is your numerical method stable for all time steps?
\item Is your numerical method conservative?
\item Is there any spurious behaviour in your solution? (eg unbounded results
or unrealistic oscillations).
\item Compare explicit and implicit time-stepping.
\item You could consider using Runge-Kutta time-stepping
\item You could increase the order of accuracy of the spatial discretisation
\item If you are solving the shallow water equations
\begin{enumerate}
\item Try adding non-linear terms
\item Try using a different Arakawa grid
\end{enumerate}
\item Try using variable spatial resolution or adaptive time-stepping
\end{enumerate}
You should search online for ideas of a numerical method to use. You
can also refer to Hilary's MSc teaching notes:\\
{\small{} \url{http://www.met.reading.ac.uk/~sws02hs/teaching/MTMFMD/MTMFMD_numerics_lecturer_2.pdf}}{\small\par}
