
\chapter{Software Architecture}

There are numerous very bad programming practices in science. We will
cover a few ways in which things could be done better. 

\section{Some Good Programming Practices}\label{sec:goodCode}

\begin{enumerate}
\item Plan your code before writing it. Work out exactly what variables you will need and how big all the arrays will need to be. Work out how to structure your code using data structures and functions.
\item Comments:
    \begin{enumerate}
    \item All codes should have over-arching comments, ie general comments describing the overall purpose of the whole programme. 
    \item All functions should have doc-strings. Doc-strings are descriptions of the function to document them. They are in double quotes and appear immediately after the function definition line. 
    \item Blocks of code and loops should have comments describing what they do.
    \item Do not make obvious comments (eg loop from 0 to $n-1$ every 1).
    \item Make sure that comments are up to date with the code.
    \end{enumerate}

\item Write efficient code:
    \begin{enumerate}
    \item Do not declare more arrays than you need (unless this dramatically improves the style).
    \item Where possible, avoid conditionals inside loops. For example, do not test a conditional every time around a loop if you know exactly where it will be different.
    
     {\bf Good \hspace{-0.05\linewidth}}
    \begin{minipage}[t]{0.45\linewidth}\vspace{-10pt}
        \begin{lstlisting}
        u[0] = ...
        u[N] = ...
        for i in xrange(1,N):
            u[i] = ...
        \end{lstlisting}
    \end{minipage}
    {\bf Bad \hspace{-0.05\linewidth}}
    \begin{minipage}[t]{0.3\linewidth}\vspace{-10pt}
        \begin{lstlisting}
        for i in xrange(0,N+1):
            if i == 0:
                u[i] = ...
            else if i == N:
                u[i] = ...
            else:
                u[i] = ...
        \end{lstlisting}
    \end{minipage} \vspace{-12pt} \\
The second version is more complex code, longer and more expensive to run.
    \item When looping over an array, do not recalculate the whole array every time around.
    \end{enumerate}

\item Coding to Avoid Errors:
    \begin{enumerate}
    \item Avoid code duplication by using functions and by getting the structure of the code right.

    \item Avoid very deep nested loops. Use functions instead.

    \item Use variables rather than having input parameters as numbers and pass variables as arguments to functions rather than re-defining them in each function.

    \item Use data structures to avoid long argument lists to functions.

    \item If there are any dependencies between variables, calculate these in the code.
    
    \item Avoid global variables apart from for true constants such as $\pi$ and $e$.
    \end{enumerate}

\item Code readability:
    \begin{enumerate}
    \item Put space between functions, space between blocks, space either side of $=$ and space where needed to make the code more readable. For example:\\
    \begin{minipage}[t]{0.35\linewidth}\centerline{
        \tt y = a*x**2 + b*x + c
    }\end{minipage}
    \begin{minipage}[t]{0.25\linewidth}\centerline{
        not
    }\end{minipage}
    \begin{minipage}[t]{0.35\linewidth}\centerline{
        \tt y=x ** 2+b*x+c
    }\end{minipage}
    
    \item Do not use more brackets than you need.

    \item Wrap long lines -- unless inside brackets, use \textbackslash\ at the end of the line so that the next line is a continuation. Set up your text editor to avoid lines longer than 80 characters.
    
    \item Make the code look like the maths.
    
    \item Use good variable names (eg use {\tt i} to loop over an array {\tt x} in direction $x$.)

    \end{enumerate}
    
    \item Re-write. It is difficult to get the design right first time. Once you have got it all working, now is the time to re-structure to improve readability, efficiency and extensibility. 
    
    \item Python specific: If you are new to using arrays and looping over arrays you should make sure that you can do all initialisation and manipulation of arrays by hand, looping over arrays, rather than using inbuilt python handling of arrays. However once you are confident manipulating arrays, you should use numpy array manipulations instead of explicit loops as this will be faster and take fewer lines of code.
\end{enumerate}




\section{Testing\label{sec:testing}}

\paragraph*{Important for this week}

Code should be tested in situations where you know exactly what the
output should. This is a crucial part of writing bug-free code. Every
code and every function within a code should be tested. For a function,
you should think of a variety of different inputs for which you know
the output and write a tests that calls the function with those inputs
and checks that the outputs are as expected. There may also be situations
where a function is expected to give an error message. The tests should
include input that you know should give an error message.

\paragraph*{Further information}

There are python modules to help with testing such as \url{unittest}
and \url{nosetest}. Tools like \url{coverage} can show you how much
of your code you have tested. There is good integration with some
integrated development environments (IDEs), such as \url{PyCharm},
which helps with most aspects of writing, testing, debugging, and
running code. 

\clearpage{}

\stepwise{

\section{Debugging}

\paragraph*{Important for this week}

You should be able to debug by hand without a dedicated tool, using
print statements:
\begin{itemize}
\item \step{Re-produce the problem with the simplest code and with arrays
set as small as possible so that the code runs quickly and so that
you can re-produce the calculations by hand.}\step{
\item You need to know where things go wrong. Start by putting in print
statements at a point in the code shortly before everything went wrong.
Print out variables that you think might be related to the problem
and check that their values are as expected. }\step{
\item Put in print statements further up where you think that everything
was working fine, printing out relevant variables.}\step{
\item Using more print statements, find where in between things go wrong.
}\step{
\item When you have found and corrected the bug, delete the de-bugging code
rather than commenting it out. You do not want your code littered
with commented out code. You could save the debug statements in a
branch of your repository. }\step{
\item Think how this bug happened and how you can better test your code
to avoid similar bugs.}
\end{itemize}

\paragraph*{\pause Further Information}

IDEs (Interactive Development Environments, such as PyCharm) have
useful interactive debugging tools which allow you to set breakpoints
in the code and see, for example, what all variable values are. There
are also python libraries such as \url{pdb} which stop the code at
the line \url{pdb.set_trace()} and then allow you to step through
and see what all variable values are. }\clearpage{}

\section{Version Control using git\label{sec:git}}

Very quickly after starting to write code, you will find that you
have modified the code and it doesn't do what you want any more, you
have multiple versions and you can't keep track, two people are working
on the same code and getting in a muddle, you cannot reproduce the
results that you got yesterday. You need to be using version control
software like \url{git} and a repository hosting service like \url{https://github.com/},
which also provides archiving and backup.

\pause There are numerous online resources for how to use \url{git}
and \url{github}. This will therefore be a tiny introduction. A more
comprehensive introduction is at, for example:\\
\url{https://swcarpentry.github.io/git-novice/}\\
and\\
{\small{}\url{http://product.hubspot.com/blog/git-and-github-tutorial-for-beginners}}.

You should really also know about collaborating, branching and merging
in \url{git} but this little introduction will cover setting up a
repository, adding to it, pushing and pulling from a remote repository
and recovering old versions. This introduction draws heavily from
the Software Carpentry introduction. \pause 

\subsection{Setting up \protect\url{git}}

These are the commands that I used to set up \url{git} for my login
on the machines at Cambridge:

\begin{lstlisting}
$ git config --global user.name "Hilary Weller"
$ git config --global user.email "h.weller@reading.ac.uk"
$ git config --global color.ui "auto"
$ git config --global core.editor "gedit --wait --new-window"
\end{lstlisting}

\clearpage{}

\subsection{Setting up a Local Repository}

Next set up a local repository for my \url{helloWorld.py} code:

\begin{lstlisting}
# Initialise the local directory as a git repository
$ git init
# Add files to the repo, check and commit them with a commit message
$ git add helloWorld.py
# Now the file is staged for commit. This can be checked and committed:
$ git status
$ git commit -m "First commit"
\end{lstlisting}
\pause 

\subsection{Sign up for a \protect\url{github} account and create a Remote Repository}
\begin{itemize}
\item Sign up for a free account allowing unlimited public code at \url{https://github.com/}
\item Create a new repository at \url{https://github.com/new} or by following
the links. I created a repository called \url{helloWorld}. My new
\url{github} user name is \url{studentSpencer0}. 
\end{itemize}
\pause 

\subsection{Link the local and remote repositories}

Copy the url of your new repository. Mine is \url{https://github.com/studentSpencer0/helloWorld}.
Then, on the command line:
\begin{lstlisting}
$ git remote add origin https://github.com/studentSpencer0/helloWorld
$ git remote -v     # (to verify the new remove URL)
\end{lstlisting}
\pause 

\subsection{Push the changes in your local repository to GitHub.}

If you initialised the remote repository with a README pull and merge
before push
\begin{lstlisting}
$ git pull origin master
\end{lstlisting}
After closing the window that asks you to write a merge message you
can now do:

\begin{lstlisting}
$ git push origin master
\end{lstlisting}
 Now have a look at the url for your repository and see your new code
there.

\clearpage{}

Once everything is set up you can push and pull code using:
\begin{lstlisting}
$ git pull
$ git push
\end{lstlisting}
You can avoid entering your \url{github} username and password every
time you push by typing:
\begin{lstlisting}
git config credential.helper store
\end{lstlisting}
\pause 

\subsection{Adding all suitable files}

We may have a directory structure full of files that we want to add
to the repository. However \textcolor{blue}{it is important not to
add large or binary files to a git repository}. Therefore we need
a \url{.gitignore} file to tell git not to add files with well known
extensions that imply that they are binary files. You can copy my
\url{.gitignore} file:

\begin{lstlisting}[basicstyle=\scriptsize]
$ wget https://raw.githubusercontent.com/studentSpencer0/helloWorld/master/.gitignore
\end{lstlisting}

You can now add all files in the directory structure to the repository
without the risk of adding large or binary files
\begin{lstlisting}
$ git add .
\end{lstlisting}
\clearpage{}

\subsection{Retrieving Old Versions}
\begin{itemize}
\item In order to see a list of commit ids and messages:
\begin{lstlisting}
$ git log
\end{lstlisting}
\pause 
\item If you make some changes to a file (say \url{helloWorld.py}), you
can see the differences between what is on disk and the version of
the previous commit:
\begin{lstlisting}
$ git diff helloWorld.py
# or if you have already added helloWorld.py but not committed:
$ git diff HEAD helloWorld.py
# or if you want to compare helloWorld.py to the version from 
# three commits ago:
$ git diff HEAD~3 helloWorld.py
\end{lstlisting}
\pause 
\item You can also show the difference between a file on disk and a previous
commit using the commit id:
\begin{lstlisting}
$ git diff cb27fc3e526dbdcd49b70b3f08a520ae271f8be5 helloWorld.py
\end{lstlisting}
\pause 
\item To show what changes were made at an older commit rather than differences
between working directory and a commit:
\begin{lstlisting}
$ git show HEAD~3 helloWorld.py
\end{lstlisting}
\pause 
\item If you need to discard the changes that you have made and go back
to the most recent commit:
\begin{lstlisting}
$ git checkout -- helloWorld.py
\end{lstlisting}
or to go back to a particular commit (just using first few letters
of the commit id):
\begin{lstlisting}
$ git checkout f22b25e helloWorld.py
\end{lstlisting}
\end{itemize}

