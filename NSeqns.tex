
\chapter{The Navier Stokes Equations}

The Navier-Stokes Equations for a compressible, rotating atmosphere

{\global\long\def\arraystretch{0.8}

\begin{tabular}{ll}
Momentum  & $\frac{D\mathbf{u}}{Dt}=\text{-}2\bm{\Omega}\times\mathbf{u}-\frac{\nabla p}{\rho}+\mathbf{g}+\mu_{u}\left(\nabla^{2}\mathbf{u}+\frac{1}{3}\nabla(\nabla\cdot\mathbf{u})\right)$ \tabularnewline
 & \tabularnewline
Continuity  & $\frac{D\rho}{Dt}+\rho\nabla\cdot\mathbf{u}=0$\tabularnewline
 & \tabularnewline
Potential temperature  & $\frac{D\theta}{Dt}=Q+\mu_{\theta}\nabla^{2}\theta$\tabularnewline
 & \tabularnewline
\multicolumn{2}{l}{An equation of state, eg perfect gas law, $p=\rho RT$}\tabularnewline
 & \tabularnewline
\multicolumn{2}{l}{Where the Lagrangian derivative is defined as $\frac{D\Psi}{Dt}=\frac{\partial\Psi}{\partial t}+\mathbf{u}\cdot\nabla\Psi$}\tabularnewline
\end{tabular}}{\global\long\def\arraystretch{1}

\begin{flushleft}
\begin{tabular}{llll}
$\mathbf{u}$  & Wind vector  & $\vec{g}$  & Gravity vector (downwards)\tabularnewline
$t$  & Time  & $\theta$  & Potential temperature, $T\left(p_{0}/p\right)^{\kappa}$\tabularnewline
$\bm{\Omega}$  & Rotation rate of planet & $\kappa$ & heat capacity ratio $\approx1.4$\tabularnewline
$\rho$  & Density of air & $Q$ & Source of heat\tabularnewline
$p$  & Atmospheric pressure & $\mu_{u}$, $\mu_{\theta}$  & Diffusion coefficients\tabularnewline
\end{tabular}
\par\end{flushleft}

}\pause 
\begin{itemize}
\item What does $\nabla\cdot\mathbf{u}$ mean?\pause 
\item What does $\nabla p$ mean?\pause 
\item What does $\mathbf{u}\cdot\nabla\Psi$ mean?\pause 
\item What does $\bm{\Omega}\times\mathbf{u}$ mean?
\end{itemize}

\section{Advection and Diffusion\label{sec:Advection-of-Pollution}}
\begin{center}
\begin{tabular}{ccccccccc}
$\frac{D\Psi}{Dt}$ & $=$ & $\frac{\partial\Psi}{\partial t}$ & $+$ & $\mathbf{u}\cdot\nabla\Psi$ & $=$ & $S$ & $+$ & $\mu_{\Psi}\nabla^{2}\Psi$\tabularnewline
Lagrangian &  & Rate of change &  & Advection &  & Sources &  & Diffusion\tabularnewline
derivative &  & at fixed point &  & of $\Psi$ &  & and sinks &  & of $\Psi$\tabularnewline
\end{tabular}
\par\end{center}

\mediaMovie[autostart]{\includegraphics[width=0.8\textwidth]{/home/hilary/Videos/pollutionPlumes/pollutionPlumes-still.png}}{/home/hilary/Videos/pollutionPlumes/pollutionPlumes.avi}

\section{The Momentum Equation\label{sec:momEqn}}
\begin{center}
\begin{tabular}{ccccccccc}
$\frac{D\mathbf{u}}{Dt}$ & $=$ & $\text{-}2\bm{\Omega}\times\mathbf{u}$ & - & $\frac{\nabla p}{\rho}$ & + & $\mathbf{g}$ & $+$ & $\mu_{u}\left(\nabla^{2}\mathbf{u}+\frac{1}{3}\nabla(\nabla\cdot\mathbf{u})\right)$\tabularnewline
Lagrangian &  & Coriolis &  & Pressure & \multicolumn{3}{c}{Gravitational} & Diffusion\tabularnewline
derivative &  &  &  & gradient & \multicolumn{3}{c}{acceleration} & \tabularnewline
\end{tabular}
\par\end{center}

\subsection{Coriolis\label{subsec:Coriolis}}

Inertial Oscillations governed by part of the momentum equation:

\begin{minipage}[c]{0.35\columnwidth}%
\[
\frac{\partial\mathbf{u}}{\partial t}=\text{-}2\bm{\Omega}\times\mathbf{u}
\]

\begin{itemize}
\item \begin{flushleft}
A drifting buoy set in motion by strong westerly winds in the Baltic
Sea in July 1969. 
\par\end{flushleft}
\item \begin{flushleft}
Once the wind subsides, the upper ocean follows inertia circles
\par\end{flushleft}
\end{itemize}
\includegraphics[width=1\linewidth]{/home/hilary/latex/teaching/topics/NSintro/copyright_persson_broman}%
\end{minipage}\hfill{}%
\begin{minipage}[c]{0.6\columnwidth}%
\includegraphics[width=1\linewidth]{/home/hilary/latex/teaching/topics/NSintro/inertia_circle}%
\end{minipage}

\subsection{The Pressure Gradient Force}

\begin{minipage}[c]{0.48\columnwidth}%
If the pressure gradient force is the only large term in the momentum
equation, then together with the continuity equation and perfect gas
law, we get equations for acoustic waves:
\begin{eqnarray*}
\frac{\partial\mathbf{u}}{\partial t}+\frac{1}{\rho_{0}}\nabla p & = & 0\\
\frac{\partial p}{\partial t}+\rho_{0}c^{2}\nabla\cdot\mathbf{u} & = & 0
\end{eqnarray*}
where $\rho_{0}$ is a reference density and $c$ is the speed of
sound.%
\end{minipage}\hfill{}%
\begin{minipage}[c]{0.48\columnwidth}%
\mediaMovie[autostart,loop]
{\includegraphics[width=\textwidth]{/home/hilary/latex/teaching/topics/NSintro/Spherical_wave/Spherical_wave2-0.jpg}}
{/home/hilary/latex/teaching/topics/NSintro/Spherical_wave/Spherical_wave2.avi}%
\end{minipage}

\clearpage{}

\subsection*{Pressure Gradients lead to very fast acceleration - Acoustic Waves}

\mediaMovie[autostart=true,loop]{\includegraphics[width=0.8\textwidth]{/home/hilary/Videos/pressureWaves/bombBlast.jpg}}{/home/hilary/Videos/pressureWaves/bombBlast.avi}

\clearpage{}

\subsection*{Geostrophic Balance: Pressure Gradients versus Coriolis}

$\text{-}2\bm{\Omega}\times\mathbf{u}=\frac{\nabla p}{\rho}$ . If
$2\bm{\Omega}=\begin{pmatrix}0\\
0\\
f
\end{pmatrix}$ and $\mathbf{u}=\begin{pmatrix}u\\
v\\
w
\end{pmatrix}$ then $u=\optBlock{-\frac{1}{f\rho}\frac{\partial p}{\partial y}}$,
$v=\optBlock{\frac{1}{f\rho}\frac{\partial p}{\partial x}}$. 

\includegraphics[height=0.8\textheight]{/home/hilary/latex/teaching/topics/NSintro/synnopticChart}

\clearpage{}

\subsection*{Geostrophic turbulence: pressure gradients, Coriolis and the (non-linear)
advection of velocity by velocity}

\mediaMovie[autostart]{\includegraphics[width=0.8\textwidth]{/home/hilary/Videos/geoTurbulence/geoTurbulence.jpg}}{/home/hilary/Videos/geoTurbulence/geoTurbulence.avi}

\subsection{Gravitational Acceleration: Explosive Comulonimbus}

\mediaMovie[autostart]{\includegraphics[width=0.8\textwidth]{/home/hilary/Videos/convectivePlumes/explosiveComulonumbus.jpg}}{/home/hilary/Videos/convectivePlumes/explosiveComulonumbus.avi}

\clearpage{}

\subsection{The Complete Navier Stokes Equations}

With moisture, phase changes, radiation, ... from NUGAM (courtesy
of Pier Luigi)

\mediaMovie[autostart,loop]
{\includegraphics[width=\textwidth]{/home/hilary/Videos/NUGAM/NUGAM_Sun_1yr_web.jpg}}
{/home/hilary/Videos/NUGAM/NUGAM_Sun_1yr_web.avi}
