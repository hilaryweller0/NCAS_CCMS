
\chapter{Time Stepping for Atmospheric Models}

\section*{Aim}
\begin{itemize}
\item Describe some time stepping methods used in atmospheric modelling
and how they relate to each other
\item Describe their advantages and disadvantages
\item What modelling centres use what methods
\end{itemize}

\section*{Time Stepping and Related Methods}

\begin{multicols}{2}
\begin{itemize}
\item Runge-Kutta or multi-step
\item Implicit or explicit
\item Semi-implicit, split explicit or \\
HEVI \textendash{}\uline{ H}orizontally \uline{E}xplicit, \uline{V}ertically
\uline{I}mplicit
\item Projection method
\item Elliptic or hyperbolic equations
\item Newton's method (not described)
\item Exponential integrators (not described)
\item Parallel in time (not described)
\end{itemize}
\end{multicols}

\clearpage{}

\section{Runge-Kutta (RK)}

A Runge-Kutta or multi-stage scheme uses two time levels and a number
of intermediate steps to go from time step $n$ to $n+1$. Here are
some well known RK schemes in order to solve the ODE:
\[
{\displaystyle \frac{d\mathbf{y}}{dt}=\mathbf{f}(\mathbf{y})}
\]

\begin{lyxlist}{00.00.0000}
\item [{Forward-backward}] $\begin{array}[t]{rcl}
\mathbf{y}^{\prime} & = & \mathbf{y}^{n}+\Delta t\mathbf{f}(\mathbf{y}^{n})\\
\mathbf{y}^{n+1} & = & \mathbf{y}^{n}+\Delta t\mathbf{f}(\mathbf{y}^{\prime})
\end{array}$
\item [{Fourth-order}] $\begin{array}[t]{rcl}
\mathbf{y}_{1} & = & \mathbf{y}^{n}\\
\mathbf{y}_{2} & = & \mathbf{y}^{n}+\Delta t\dfrac{1}{2}\mathbf{f}(\mathbf{y}_{1})\\
\mathbf{y}_{3} & = & \mathbf{y}^{n}+\Delta t\dfrac{1}{2}\mathbf{f}(\mathbf{y}_{2})\\
\mathbf{y}_{4} & = & \mathbf{y}^{n}+\Delta t\mathbf{f}(\mathbf{y}_{3})\\
\mathbf{y}^{n+1} & = & \mathbf{y}^{n}+\Delta t\left(\dfrac{1}{6}\mathbf{f}(\mathbf{y}_{1})+\dfrac{1}{3}\mathbf{f}(\mathbf{y}_{2})+\dfrac{1}{3}\mathbf{f}(\mathbf{y}_{3})+\dfrac{1}{6}\mathbf{f}(\mathbf{y}_{4})\right)
\end{array}$
\end{lyxlist}

\subsection*{\clearpage{}Generalising}

A $\nu$ stage RK scheme with time step $\Delta t$, $t^{n}=n\Delta t$
and $\mathbf{y}^{n}=\mathbf{y}(t^{n})$ can be written:
\begin{eqnarray*}
\mathbf{y}^{(j)} & = & \mathbf{y}^{n}+\Delta t\sum_{\ell=1}^{\nu}a_{j\ell}\mathbf{f}\left(\mathbf{y}^{(\ell)}\right)\ \text{ for }j=1\ldots\nu\\
\mathbf{y}^{n+1} & = & \mathbf{y}^{n}+\Delta t\sum_{j=1}^{\nu}w_{j}\mathbf{f}\left(\mathbf{y}^{(j)}\right).
\end{eqnarray*}
The scheme is defined by a Butcher tableau:
\[
\begin{array}{c|c}
\mathsf{\mathbf{c}} & A\\
\hline  & \mathbf{w}^{T}
\end{array}=\begin{array}{c|cccccc}
c_{1} & a_{11} & a_{12} & \cdots & a_{1\ell} & \cdots & a_{1\nu}\\
\vdots & \vdots & \vdots &  & \vdots &  & \vdots\\
c_{j} & a_{j1} & a_{j2} & \cdots & a_{j\ell} & \cdots & a_{j\nu}\\
\vdots & \vdots & \vdots &  & \vdots &  & \vdots\\
c_{\nu} & a_{\nu1} & a_{\nu2} & \cdots & a_{\nu\ell} & \cdots & a_{\nu\nu}\\
\hline  & w_{1} & w_{2} & \cdots & w_{\ell} & \cdots & w_{\nu}
\end{array}
\]
where $c_{j}$ is the stage size and $c_{j}={\displaystyle \sum_{\ell=1}^{\nu}}a_{j\ell}$

\paragraph*{\clearpage{}Exercise:}

Write down the Butcher tableau for forward-backward and the fourth-order
RK method.

\optBlock{Forward-backward: $\begin{array}[t]{c|cc}
0 & 0 & 0\\
1 & 1 & 0\\
\hline  & 0 & 1
\end{array}$ Fourth-order: $\begin{array}[t]{c|cccc}
0 & 0 & 0 & 0 & 0\\
\half & \half & 0 & 0 & 0\\
\half & 0 & \half & 0 & 0\\
1 & 0 & 0 & 1 & 0\\
\hline  & \nicefrac{1}{6} & \nicefrac{1}{3} & \nicefrac{1}{3} & \nicefrac{1}{6}
\end{array}$}

\paragraph*{Exercise:}

Write out these schemes more simply, without a Butcher tableau
\begin{enumerate}
\item %
\begin{tabular}{>{\raggedright}p{0.29\textwidth}>{\centering}p{0.29\textwidth}>{\raggedleft}p{0.29\textwidth}}
Euler Implicit &  & $\begin{array}{c|c}
1 & 1\\
\hline  & 1
\end{array}$\tabularnewline
\end{tabular}
\item %
\begin{tabular}{>{\raggedright}p{0.29\textwidth}>{\centering}p{0.29\textwidth}>{\raggedleft}p{0.29\textwidth}}
Euler Explicit &  & $\begin{array}{c|cc}
0 & 0 & 0\\
1 & 1 & 0\\
\hline  & 1 & 0
\end{array}$\tabularnewline
\end{tabular}
\item %
\begin{tabular}{>{\raggedright}p{0.29\textwidth}>{\centering}p{0.29\textwidth}>{\raggedleft}p{0.29\textwidth}}
Crank-Nicolson & \optBlock{$\begin{array}{c}
\mathbf{y}^{n+1}=\mathbf{y}^{n}+\frac{1}{2}\Delta t\left(\mathbf{f}(\mathbf{y}^{n})+\mathbf{f}(\mathbf{y}^{n+1})\right)\end{array}$} & $\begin{array}{c|cc}
0 & 0 & 0\\
1 & \half & \half\\
\hline  & \half & \half
\end{array}$\tabularnewline
\end{tabular}
\end{enumerate}
\vfill{}
Explicit schemes have $a_{j\ell}=0$ for $j\ge\ell$.

Diagonally Implicit (DIRK) schemes have $a_{j\ell}=0$ for $j>\ell$.

\clearpage{}

\section{Mutli-step Schemes}

Multi-step schemes can be written in one line. They don't use any
intermediate stages. Instead to achieve high accuracy they use values
from more than one previous time level. 

\begin{tabular}{>{\raggedright}p{0.33\textwidth}>{\centering}p{0.33\textwidth}>{\raggedleft}p{0.33\textwidth}}
To solve  & ${\displaystyle \frac{d\mathbf{y}}{dt}=\mathbf{f}(\mathbf{y})}$ & \tabularnewline
\end{tabular}

with a $K$-step multi-step method 
\[
\mathbf{y}^{n+1}=\sum_{k=0}^{K-1}\alpha_{k}\mathbf{y}^{n-k}+\Delta t\sum_{k=0}^{K-1}\beta_{k}\mathbf{f}\left(\mathbf{y}^{n-k}\right)+\Delta t\gamma\mathbf{f}\left(\mathbf{y}^{n+1}\right).
\]

$\gamma>0\implies$ implicit, $\gamma=0\implies$ explicit 

\paragraph*{Example/Exercise}
\begin{enumerate}
\item Find $\alpha_{k}$, $\beta_{k}$, $\gamma$ for BDF2 (second-order,
backward) \optBlock{$\alpha_{0}=\nicefrac{4}{3}$, $\alpha_{1}=-\nicefrac{1}{3}$,
$\beta_{k}=0$, $\gamma=\nicefrac{2}{3}$}
\[
\mathbf{y}^{n+1}=\frac{4}{3}\mathbf{y}^{n}-\frac{1}{3}\mathbf{y}^{n-1}+\frac{2}{3}\Delta t\mathbf{f}\left(\mathbf{y}^{n+1}\right)
\]
\item Write out and name the multi-step scheme with $\alpha_{0}=0$, $\alpha_{1}=1$,
$\beta_{0}=2$, $\beta_{1}=0$, $\gamma=0$.
\[
\optBlock{\text{Leapfrog}\qquad\mathbf{y}^{n+1}=\mathbf{y}^{n-1}+2\Delta t\mathbf{f}\left(\mathbf{y}^{n}\right)}
\]
\end{enumerate}
\clearpage{}

\section{Multi-step versus multi-stage, implicit versus explicit}

\optBlock{
\begin{itemize}
\item Multi-stage is more expensive and there are usually more function
evaluations
\item Multi-step methods have computational modes (spurious solutions)
\item Implicit techniques are slower per time step because they require
the solution of a large matrix equation
\item Explicit techniques are unstable for large time steps
\end{itemize}
}

\clearpage{}

\section{Semi-implicit solution of the Shallow Water Equations}

\begin{align}
\frac{\partial\mathbf{u}}{\partial t}+\underbrace{\mathbf{u}\cdot\nabla\mathbf{u}}_{\text{advection: slow}} & =-\underbrace{2\Omega\times\mathbf{u}}_{\text{Coriolis: slow}}-\underbrace{g\nabla h}_{\text{pressure gradient: fast}}\label{eqn:SWEuv2}\\
\frac{\partial h}{\partial t}+\underbrace{\mathbf{u}\cdot\nabla h}_{\text{advection: slow}} & =-\underbrace{h\nabla\cdot\mathbf{u}}_{\text{divergence: fast}}\label{eqn:SWEh2}
\end{align}
Treat fast terms implicitly and slow terms explicitly.\stepwise{\step{

First-order Euler implicit/explicit:
\begin{eqnarray}
\frac{\mathbf{u}^{\opttext{n+1}}-\mathbf{u}^{\opttext{n}}}{\Delta t} & = & -\mathbf{u}^{\opttext{n}}\cdot\nabla\mathbf{u}^{\opttext{n}}-2\Omega\times\mathbf{u}^{\opttext{n}}-g\nabla h^{\opttext{n+1}}\label{eq:SWEuv_time}\\
\frac{h^{\opttext{n+1}}-h^{\opttext{n}}}{\Delta t} & = & -\mathbf{u}^{\opttext{n}}\cdot\nabla h^{\opttext{n}}-h^{\opttext{n}}\nabla\cdot\mathbf{u}^{\opttext{n+1}}\ \text{(note linearisation)}\label{eq:SWEh_time}
\end{eqnarray}
}\step{

Re-arrange (\ref{eq:SWEuv_time}) for $\mathbf{u}^{n+1}$, substitute
into (\ref{eq:SWEh_time}) and re-arrange to get a Helmholtz equation
for $h$, discretised in time:

}}

\clearpage{}

\stepwise{
\begin{eqnarray}
\mathbf{u}^{n+1} & = & \opttext{\mathbf{u}^{n}-\Delta t\left(\mathbf{u}^{n}\cdot\nabla\mathbf{u}^{n}+2\Omega\times\mathbf{u}^{n}+g\nabla h^{n+1}\right)}\label{eq:u_backSubs}\\
h^{n+1} & = & \opttext{h^{n}-\Delta t\mathbf{u}^{n}\cdot\nabla h^{n}-\Delta th^{n}\nabla\cdot\left(\mathbf{u}^{n}-\Delta t\left(\mathbf{u}^{n}\cdot\nabla\mathbf{u}^{n}+2\Omega\times\mathbf{u}^{n}\right)\right)}\label{eq:Helm1}\\
 & + & \Delta t^{2}gh^{n}\nabla^{2}h^{n+1}.
\end{eqnarray}

\step{If we write \vspace{-10mm}
\begin{eqnarray*}
\mathbf{u}^{\prime} & = & \mathbf{u}^{n}-\Delta t\left(\mathbf{u}^{n}\cdot\nabla\mathbf{u}^{n}+2\Omega\times\mathbf{u}^{n}\right)\\
h^{\prime} & = & h^{n}-\Delta t\left(\mathbf{u}^{n}\cdot\nabla h^{n}-h^{n}\nabla\cdot\mathbf{u}^{\prime}\right)
\end{eqnarray*}
then we can write (\ref{eq:u_backSubs}) and (\ref{eq:Helm1}) as:}
\begin{eqnarray}
 & \opttext{\mathbf{u}^{n+1}=\mathbf{u}^{\prime}-\Delta tg\nabla h^{n+1}}\label{eq:uprime_backSubs}\\
 & \opttext{h^{n+1}=h^{\prime}+\Delta t^{2}gh^{n}\nabla^{2}h^{n+1}.}\label{eq:Helm2}
\end{eqnarray}

\step{

If we assume a one dimensional C-grid ($u$ and $h$ staggered) with
$u_{j+\half}^{n}=u\left((j+\half)\Delta x,n\Delta t\right)$, $h_{j}^{n}=h(j\Delta x,n\Delta t)$
and we ignore spatial discretisation of $\mathbf{u}^{\prime}$ and
$h^{\prime}$ for now, then (\ref{eq:uprime_backSubs}) and (\ref{eq:Helm2})
can be discretised in space and time using centred differences as:}}

\clearpage{}

\stepwise{
\begin{eqnarray}
\mathbf{u}_{j+\half}^{n+1} & = & \opttext{\mathbf{u}_{j+\half}^{\prime}-\Delta tg\frac{h_{j+1}^{n+1}-h_{j}^{n+1}}{\Delta x}}\label{eq:uprime_backSubs-C}\\
h_{j}^{n+1} & = & \opttext{h_{j}^{\prime}+\Delta t^{2}gh_{j}^{n}\frac{h_{j+1}^{n+1}+h_{j-1}^{n+1}-2h_{j}^{n+1}}{2\Delta x^{2}}.}\label{eq:Helm2-C}
\end{eqnarray}

\step{Or if we use an A-grid ($u$ and $h$ co-located) with $u_{j}^{n}=u\left(j\Delta x,n\Delta t\right)$,
$h_{j}^{n}=h(j\Delta x,n\Delta t)$ and we ignore spatial discretisation
of $\mathbf{u}^{\prime}$ and $h^{\prime}$ for now, then (\ref{eq:uprime_backSubs})
and (\ref{eq:Helm2}) can be discretised in space and time using centred
differences as:}\step{
\begin{eqnarray}
\mathbf{u}_{j}^{n+1} & = & \opttext{\mathbf{u}_{j}^{\prime}-\Delta tg\frac{h_{j+1}^{n+1}-h_{j-1}^{n+1}}{2\Delta x}}\label{eq:uprime_backSubs-A}\\
h_{j}^{n+1} & = & \opttext{h_{j}^{\prime}+\Delta t^{2}gh_{j}^{n}\frac{h_{j+2}^{n+1}+h_{j-2}^{n+1}-2h_{j}^{n+1}}{8\Delta x^{2}}.}\label{eq:Helm2-A}
\end{eqnarray}
}}

\clearpage{}

\section{Projection Method \textendash{} for elliptic equations}

A projection method is like semi-implicit but in the limit that the
fast time-scale is infinitely fast and the Helmholtz equation becomes
a Poisson equation:
\begin{align}
\frac{\partial\mathbf{u}}{\partial t}+\underbrace{\mathbf{u}\cdot\nabla\mathbf{u}}_{\text{advection: slow}} & =-\underbrace{2\Omega\times\mathbf{u}}_{\text{Coriolis: slow}}-\nabla P\label{eqn:momEqn}\\
\nabla\cdot\mathbf{u} & =0\label{eqn:divuZero}
\end{align}
Take the divergence of (\ref{eqn:momEqn}) and take the rate of change
of (\ref{eqn:divuZero}) and eliminate $\partial\partial t(\nabla\cdot\mathbf{u})$
to get a Poisson equation for $P$:\stepwise{
\begin{equation}
\nabla^{2}P=\opttext{-\nabla\cdot\left(\mathbf{u}\cdot\nabla\mathbf{u}+2\Omega\times\mathbf{u}\right).}\label{eq:Poisson}
\end{equation}
If we can find $P$ to satisfy (\ref{eq:Poisson}) then discretise
(\ref{eqn:momEqn}) in time using this $P$ then $\mathbf{u}^{n+1}$
will have the same divergence as $\mathbf{u}^{n}$ (ie zero). }

\clearpage{}

\section{HEVI \textendash{}\uline{ H}orizontally \uline{E}xplicit, \uline{V}ertically
\uline{I}mplicit}

In models of the atmosphere, typically $\Delta z\ll\Delta x$ and
so time-steps due to vertically propagating acoustic waves impose
a more severe time step restriction than anything else. 

\paragraph*{Exercise}

If the Courant numbers due to horizontal and vertical advection are
$c_{u}=u\Delta t/\Delta x$ and $c_{w}=w\Delta t/\Delta z$ and due
to horizontal and vertical propagagion of acoustic waves are $c_{x}=c\Delta t/\Delta x$
and $c_{z}=c\Delta t/\Delta z$ where $c$ is the speed of sound,
estimate these Courant numbers given $u=100$m/s, $w=10$m/s, $\Delta x=1$km,
$\Delta z=100$m, $\Delta t=1$s and $c=343$m/s. 

\optBlock{

$c_{u}=0.1$, $c_{w}=0.1$, $c_{x}=0.343$, $c_{z}=3.43$

}\stepwise{\step{

Typically, explicit schemes are stable for Courant numbers less than
one and so an implicit scheme is needed for a Courant number greater
than one. $\therefore$ use a scheme that treats vertically propagating
acoustic waves implicitly. This leads to a matrix equations to solve
for every column of the mesh. The size of the matrix is equal to the
number of model layers.

}}

\clearpage{}

\section{Split Explicit}

If we want to increase the time step then we could use a semi-implicit
method (treating all acoustic waves implicitly) but this involves
a solution of a large matrix equation. Instead we could keep the model
HEVI but do 3 steps of the acoustic wave terms for every one step
of the advection terms. 

\subsubsection*{What models are split explicit, HEVI and semi-implicit?}

Semi-implicit: \optBlock{UK Met Office, ECMWF IFS and FVM, Environment Canada}

HEVI: \optBlock{DYNAMICO, ICON}

Split explicit: \optBlock{MPAS, NICAM, WRF}

Explicit (no acoustic waves) \optBlock{CAM}

\subsubsection*{Which RK and multi-step schemes are used by which models?}

\clearpage{}

\nocite{WLW13}

\nocite{GKC13}
